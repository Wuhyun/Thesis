%!TEX root = ../thesis.tex
%*******************************************************************************
%****************************** Third Chapter **********************************
%*******************************************************************************
\chapter{Bispectrum and Primordial Non-Gaussianity}

% **************************** Define Graphics Path **************************
\ifpdf
    \graphicspath{{Chapter3/Figs/Raster/}{Chapter3/Figs/PDF/}{Chapter3/Figs/}}
\else
    \graphicspath{{Chapter3/Figs/Vector/}{Chapter3/Figs/}}
\fi

\section{Bispectrum}

Bispectrum definition. Form. Different configurations. 

\section{Primordial non-Gaussianity}

\section{CMB bispectrum estimation}

Existence of non-vanishing bispectrum at the end of inflation due to primordial non-Gaussianity leaves imprints on the statistics of CMB anisotropy.

\subsection{CMB bispectrum}

Consider the three-point correlation function of the spherical harmonic coefficients $a_{lm}^X$s from (TODO: reference alm definition here).
\begin{align}
	\left< a_{l_1 m_1}^{X_1} a_{l_2 m_2}^{X_2} a_{l_3 m_3}^{X_3}  \right> = (4\pi)^3 (-i)^{l_1 + l_2 + l_3} \left< \prod_{j=1}^{3} \left[ \int \frac{d^3\vv{k}_j}{(2\pi)^3} \zeta(\vv{k}_j)   \Delta_{l_j}^{X_j} (k_j) Y^*_{l_j m_j} (\hat{\vv{k}}_j) \right] \right>. \label{eqn:bispectrum_derivation_base_form}
\end{align}
Here $X_j$s can be either $T$ or $E$, corresponding to temperature and E-mode polarisation of CMB anisotropy, respectively. The transfer functions $\Delta_l^X (k)$ depend only on $k=\left\| \vv{k} \right\|$ and incorporates all information about from the evolution of primordial perturbations $\zeta$ to projection onto the observed sky today.

We may take everything but $\zeta(\vv{k_j})$s outside the brackets $\left< \cdot \right>$. From the definition of bispectrum, we have
\begin{align}
	\left< \zeta(\vv{k}_1) \zeta(\vv{k}_2)  \zeta(\vv{k}_3) \right> &= (2\pi)^3 \delta^{(3)}(\vv{k}_1 + \vv{k}_2 + \vv{k}_3) B(k_1, k_2, k_3) \label{eqn:primordial_bispectrum}\\
	&= \int d^3 \vv{r} \; e^{-i\vv{r} \cdot (\vv{k}_1 + \vv{k}_2 + \vv{k}_3)}  B(k_1, k_2, k_3). \label{eqn:bispectrum_derivation_delta_function}
\end{align}
where an integral expression is substituted for the Dirac $\delta$-function in the second line. At the cost of introducing an extra integral, we managed to express the $\delta$-function in a separable form: $\exp(\vv{r} \cdot (\vv{k}_1+\vv{k}_2+\vv{k}_3)) = \exp(\vv{r}\cdot \vv{k}_1) \exp(\vv{r}\cdot \vv{k}_2) \exp(\vv{r}\cdot \vv{k}_3)$. Remaining exponentials are rewritten using the plane wave expansion;
\begin{align}
	e^{-i \vv{k} \cdot \vv{r}} &= \sum_{l=0}^{\infty} (2l+1) (-i)^l j_l(kr) P_l(\hat{\vv{k}} \cdot \hat{\vv{r}})  \\	
	&= \sum_{l=0}^{\infty} \sum_{m=-l}^{l} 4\pi (-i)^l j_l(kr) Y_{lm}(\hat{\vv{k}}) Y^*_{lm}(\hat{\vv{r}}). \label{eqn:bispectrum_derivation_rayleigh}
\end{align}
Legendre polynomial $P_l(\hat{\vv{k}} \cdot \hat{\vv{r}})$ has been expanded using the spherical harmonic addition theorem on the last line. Now $\vv{k}$ and $\vv{r}$ mix only through their amplitudes within spherical bessel functions $j_l(kr)$. Once substituted into (\ref{eqn:bispectrum_derivation_base_form}), we can perform the angular integral $d^2 \hat{\vv{k}_j}$ separately for each $j=1,2,3$, since $d^3\vv{k}_j = dk_j k_j^2 d^2 \hat{\vv{k}_j}$. Note also that the spherical harmonic orthogonality relation is given by
\begin{align}
	\int d^2 \hat{\vv{n}} \; Y_{lm}(\hat{\vv{n}}) Y^*_{l'm'}(\hat{\vv{n}}) = \delta_{l l'} \delta_{m m'}.
\end{align}
Incorporating (\ref{eqn:bispectrum_derivation_base_form}), (\ref{eqn:bispectrum_derivation_delta_function}, and (\ref{eqn:bispectrum_derivation_rayleigh}), we obtain
\begin{align}
	&\left< a_{l_1 m_1}^{X_1} a_{l_2 m_2}^{X_2} a_{l_3 m_3}^{X_3}  \right> \nonumber \\
	&\hspace{0.05\textwidth}= \frac{(4\pi)^3}{(2\pi)^9} (-1)^{l_1 + l_2 + l_3} \int d^3 \vv{r} \; d^3 \vv{k}_1 d^3 \vv{k}_2 d^3 \vv{k}_3 \; B(k_1,k_2,k_3) \nonumber \\
	&\hspace{0.25\textwidth} \times \prod_{j=1}^{3} \left[ \sum_{l'_j = 0}^{\infty} \sum_{m'_j=-l'_j}^{l'_j} j_{l'_j} (k_j r) Y_{l'_j m'_j} (\hat{\vv{k}}_j) Y^*_{l'_j m'_j} (\hat{\vv{r}}) \Delta_{l_j}^{X_j} (k_j) Y^*_{l_j m_j} (\hat{\vv{k}}_j) \right]  \label{eqn:bispectrum_derivation_main_line1}\\
	&\hspace{0.05\textwidth}= \left( \frac{2}{\pi} \right)^3 \int d^3 \vv{r} \; dk_1 dk_2 dk_3 \left( k_1 k_2 k_3 \right)^2 B(k_1, k_2, k_3) \prod_{j=1}^{3} \left[ j_{l_j} (k_j r) Y^*_{l_j m_j} (\hat{\vv{r}}) \Delta_{l_j}^{X_j} (k_j) \right]  \label{eqn:bispectrum_derivation_main_line2}\\
	&\hspace{0.05\textwidth}= \left( \frac{2}{\pi} \right)^3 \mathcal{G}^{l_1 l_2 l_3 *}_{m_1 m_2 m_3} \int dr \; dk_1 dk_2 dk_3 r^2 \left( k_1 k_2 k_3 \right)^2 B(k_1, k_2, k_3) \prod_{j=1}^{3} \left[ j_{l_j} (k_j r) \Delta_{l_j}^{X_j} (k_j) \right], \label{eqn:bispectrum_derivation_main_line3}
\end{align}
where the Gaunt integral is defined as
\begin{align}
	\mathcal{G}^{l_1 l_2 l_3}_{m_1 m_2 m_3} := \int d^2 \hat{\vv{n}} \; Y_{l_1 m_1} (\hat{\vv{n}}) Y_{l_2 m_2} (\hat{\vv{n}}) Y_{l_3 m_3} (\hat{\vv{n}}). \label{def:gaunt_integral}
\end{align}
This value is always real, so we may omit the complex conjugate in (\ref{eqn:bispectrum_derivation_main_line3}). Note also that we dropped a factor of $(-1)^{l_1+l_2+l_3}$ in (\ref{eqn:bispectrum_derivation_main_line2}). This is due to parity reasons. Spherical harmonics have definite parity; $Y_{lm}(-\hat{\vv{n}}) = (-1)^l Y_{lm}(\hat{\vv{n}})$. Applying parity transformation to the integral in (\ref{def:gaunt_integral}) gives $\mathcal{G}^{l_1 l_2 l_3}_{m_1 m_2 m_3} = (-1)^{l_1+l_2+l_3} \mathcal{G}^{l_1 l_2 l_3}_{m_1 m_2 m_3}$. The Gaunt integral therefore evaluates to zero unless $l_1+l_2+l_3$ is even.

We define the \textit{reduced} bispectrum as
\begin{align}
	b^{X_1 X_2 X_3}_{l_1 l_2 l_3} := \left( \frac{2}{\pi} \right)^3 \int dr dk_1 dk_2 dk_3 \left(r k_1 k_2 k_3 \right)^2 B(k_1, k_2, k_3) \prod_{j=1}^{3} \left[ j_{l_j} (k_j r) \Delta_{l_j}^{X_j} (k_j) \right]. \label{def:reduced_bispectrum}
\end{align}
The late-time bispectrum can now be written in a concise form;
\begin{align}
	\left< a_{l_1 m_1}^{X_1} a_{l_2 m_2}^{X_2} a_{l_3 m_3}^{X_3}  \right> = \mathcal{G}^{l_1 l_2 l_3}_{m_1 m_2 m_3} b^{X_1 X_2 X_3}_{l_1 l_2 l_3}. \label{eqn:late_time_bispectrum_form}
\end{align}

Recall that the three point function in (\ref{eqn:primordial_bispectrum}) is given by a product of delta function enforcing $\vv{k}_1 + \vv{k}_2 + \vv{k}_3 = \vv{0}$ and the primordial bispectrum $B(k_1,k_2,k_3)$. Its spherical harmonic counterpart (\ref{eqn:late_time_bispectrum_form}) has an analogous form. The Gaunt integral $\mathcal{G}^{l_1 l_2 l_3}_{m_1 m_2 m_3}$ contains all the geometrical information, enforcing triangle condition on $l_1$, $l_2$, $l_3$ and angular momentum conservation $m_1+m_2+m_3=0$. Meanwhile, reduced bispectrum encodes statistical information about the underlying three point functions, just like $B(k_1,k_2,k_3)$.

Value of the Gaunt integral is best represented using Wigner 3-j symbols as
\begin{align}
	\mathcal{G}^{l_1 l_2 l_3}_{m_1 m_2 m_3} = \sqrt{\frac{(2l_1+1)(2l_2+1)(2l_3+1)}{4\pi}} \begin{pmatrix}	l_1 & l_2 & l_3 \\ m_1 & m_2 & m_3 \end{pmatrix} \begin{pmatrix}	l_1 & l_2 & l_3 \\ 0 & 0 & 0 \end{pmatrix}.
\end{align}
Wigner 3-j symbols, written here as a 2-by-3 matrix, are closely related to addition of angular momenta. They are real coefficients appearing in the expansion of zero total angular momentum state $|0 \; 0\rangle$;
\begin{align}
	| 0 \; 0 \rangle = \sum_{l_1,m_1} \sum_{l_2,m_2} \sum_{l_3,m_3} \begin{pmatrix}	l_1 & l_2 & l_3 \\ m_1 & m_2 & m_3 \end{pmatrix} | l_1 m_1 \rangle | l_2 m_2 \rangle | l_3 m_3 \rangle.
\end{align}
For further details on Wigner 3-j symbols see, e.g., \cite{Olver2010nist}. We quote the following two identities for our purposes.
\begin{align}
	\sum_{m_1,m_2,m_3} { \begin{pmatrix}	l_1 & l_2 & l_3 \\ m_1 & m_2 & m_3 \end{pmatrix} }^2 &= 1, \label{eqn:wigner_3j_normalisation} \\
	{ \begin{pmatrix}	l_1 & l_2 & l_3 \\ 0 & 0 & 0 \end{pmatrix} }^2 &= \frac{1}{2} \int_{-1}^{1} d\mu \; P_{l_1}(\mu) P_{l_2}(\mu) P_{l_3}(\mu). \label{eqn:wigner_3j_legendre_integral} 
\end{align}
The normalisation condition (\ref{eqn:wigner_3j_normalisation}) can be easily derived by computing norm of the state $|0 \; 0 \rangle$ in the definition. The second identity (\ref{eqn:wigner_3j_legendre_integral}) allows us to replace a square of any given 3-j symbol satisfying $m_1=m_2=m_3=0$ with a separable integral.

We make one last definition which will prove to be useful in the next section;
\begin{align}
	h^2_{l_1 l_2 l_3} :=& \sum_{m_1, m_2, m_3} \left( \mathcal{G}^{l_1 l_2 l_3}_{m_1 m_2 m_3} \right)^2  \label{def:h2_using_gaunt_integral}\\
	=& \frac{(2l_1+1)(2l_2+1)(2l_3+1)}{4\pi} { \begin{pmatrix}	l_1 & l_2 & l_3 \\ 0 & 0 & 0 \end{pmatrix} }^2 \\
	=& \frac{(2l_1+1)(2l_2+1)(2l_3+1)}{8\pi} \int_{-1}^{1} d\mu \; P_{l_1}(\mu) P_{l_2}(\mu) P_{l_3}(\mu). \label{def:h2_using_legendre_integral}
\end{align}

By squaring the Gaunt integral and summing over $m$s, we get a simpler quantity $h^2_{l_1 l_2 l_3}$ which preserves important geometrical information. Here $l_1$, $l_2$ and $l_3$ must still form a triangle and add up to an even number. Otherwise, the integral over Legendre polynomials in (\ref{def:h2_using_legendre_integral}) vanishes.

\subsection{Optimal estimator}
Bispectrum of the CMB anisotropy is a powerful probe of primordial non-Gaussianity. Planck collaboration's CMB bispectrum analyses provided the most stringent bounds on the amplitude of primordial bispectrum of various shapes, constraining a wide class of inflationary models. Thanks to the largely linear evolution of CMB, there are little contributions to the bispectrum from non-primordial origins compared to other probes such as the large scale structure. In this section, we derive the optimal CMB bispectrum estimator using the mathematical formulation laid out previously.

Consider a number of inflation models which predict non-vanishing primordial bispectrum of form $B^{(i)}(k_1,k_2,k_3)$ for $i=1,2,\cdots$. We often use templates capturing the common characteristics of a class of models, such as local, equilateral, and orthogonal shapes (TODO: define these earlier and reference them here). We would like to find out if the true underlying bispectrum, if any, can be expanded in terms of shapes chosen;
\begin{align}
	B(k_1,k_2,k_3) = \sum_i f_{NL}^{(i)} \; B^{(i)}(k_1,k_2,k_3), \label{eqn:primordial_bispectrum_fNLs}
\end{align}
where the primordial non-Gaussianity (or non-linearity \footnote{This name originates from local non-Gaussianity, which was one of the first types of models studied. In a local model, the metric potential is expanded as a local function of a Gaussian field as $\Phi(\vv{x}) = \phi(\vv{x}) + f_{NL} \left(\phi^2(\vv{x}) - \left< \phi^2 \right> \right) + \cdots$. Here, $f_{NL}$ parametrises the amplitude of non-linear contribution to the field, hence the name `non-linearity' parameter.}) parameter $f^{(i)}_{NL}$ measures the magnitude of given bispectrum shape present in reality. Detection of non-zero $f_{NL}$ can serve as a strong evidence for the corresponding models. Non-detection of $f_{NL}$, on the other hand, still allows us to place bounds on the number and hence constrain models which predict larger bispectra.

Expressing (\ref{eqn:primordial_bispectrum_fNLs}) in terms of the late-time CMB anisotropies,
\begin{align}
	\left< a_{l_1 m_1}^{X_1} a_{l_2 m_2}^{X_2} a_{l_3 m_3}^{X_3}  \right> = \sum_i f^{(i)}_{NL} \; \mathcal{G}^{l_1 l_2 l_3}_{m_1 m_2 m_3} b^{X_1 X_2 X_3, (i)}_{l_1 l_2 l_3}. \label{eqn:late_time_bispectrum_fNLs_theoretical}
\end{align}
The goal of CMB bispectrum estimation is to compute $f^{(i)}_{NL}$s which best describes the observations. In reality, we can only observe one realisation of the universe and therefore a single set of $a_{lm}$s. Sample estimates replace the expectation values $\left< \cdot \right>$ on the left hand side of \eqref{eqn:late_time_bispectrum_fNLs_theoretical}, and we introduce the error term;
\begin{align}
	a_{l_1 m_1}^{X_1} a_{l_2 m_2}^{X_2} a_{l_3 m_3}^{X_3} = \sum_i  f^{(i)}_{NL} \; \mathcal{G}^{l_1 l_2 l_3}_{m_1 m_2 m_3} b^{X_1 X_2 X_3, (i)}_{l_1 l_2 l_3} \;+\; \epsilon^{X_1 X_2 X_3}_{l_1 l_2 l_3, m_1 m_2 m_3}. \label{eqn:late_time_bispectrum_fNLs_sample}
\end{align}
For simplicity, we drop the $X_j$'s from now on. The following derivation of the bispectrum estimator can be easily generalised to include both temperature and E-mode polarisation. We also define some shorthand notations for harmonic indices to improve readability;
\begin{align}
	\vv{l}_j:=(l_j,m_j), \;\; L:=(l_1,l_2,l_3), \;\; \text{and} \;\; \vv{L}:=(\vv{l}_1, \vv{l}_2, \vv{l}_3).
\end{align}
The estimation problem is summarised as follows.
\begin{align}
	&B^{obs}_\vv{L} = \sum_i B^{(i)}_\vv{L} f^{(i)}_{NL}  \;+\; \epsilon_\vv{L}, \label{eqn:bispectrum_estimation_core}\\
	\text{where} \;\; &B^{obs}_\vv{L} := a_{\vv{l}_1} a_{\vv{l}_2} a_{\vv{l}_3} \;\; \text{and} \;\; B^{(i)}_\vv{L} := \mathcal{G}_\vv{L} b^{(i)}_L.
\end{align}
The form of \eqref{eqn:bispectrum_estimation_core} makes it clear that bispectrum estimation is linear regression in essence. Borrowing terms from statistics, we specify the components of our analysis below.
\begin{itemize}
	\item $\vv{B}^{obs}$ is the \textit{regressand}, in our case the noisy observed bispectrum samples $B^{obs}_\vv{L}$ obtained for each \textit{observation} $\vv{L}$.
	\item $\vv{B}^{(i)}$s are the \textit{regressors}, theoretical bispectra $B^{(i)}_\vv{L}$ motivated from inflationary models.
	\item $f^{(i)}_{NL}$s are the \textit{regression coefficients} which parametrise how much each regressor contributes to the regressand. Our main objective is to estimate them.
	\item $\vv{\epsilon}$ is the \textit{error term} which represents the noise in bispectrum sample and the true underlying value. $\epsilon_\vv{L}$ can be sourced by sampling error (from limited number of samples) or inaccuracies in the measurements. 
\end{itemize}

We adopt the least squares method to estimate $f_{NL}$s from given $\vv{B}^{obs}$ and $B^{(i)}_\vv{L}$. Before doing so, each element of the regressand and regressors needs to be normalised so that the expected variances in error are uniform across observations $\vv{L}$. We compute the theoretical variance in $B^{obs}_\vv{L}$ under two assumptions. First, we in the weak non-Gaussian limit where dominant contributions to correlation functions come from the Gaussian part of $a_\vv{l}$s . Wick's theorem then reduces the problem down to summing over all possible contractions. Second, we neglect the inaccuracies in measuring $a_{lm}$s, since their contributions to total error is much smaller compared to sampling errors.

Under these assumptions, expected covariance of the observed bispectra is given by
\begin{align}
	\left< B^{obs}_\vv{L} B^{obs}_{\vv{L}'} \right> &= \left< a_{\vv{l}_1} a_{\vv{l}_2} a_{\vv{l}_3} a_{\vv{l}'_1} a_{\vv{l}'_2} a_{\vv{l}'_3} \right> 
	\label{eqn:bispectrum_variance_contractions1} \\
	&= \left[ \left< a_{\vv{l}_1} a_{\vv{l}'_1} \right> \left< a_{\vv{l}_2} a_{\vv{l}'_2} \right> \left< a_{\vv{l}_3} a_{\vv{l}'_3} \right> + \left< a_{\vv{l}_1} a_{\vv{l}'_1} \right> \left< a_{\vv{l}_2} a_{\vv{l}'_3} \right> \left< a_{\vv{l}_3} a_{\vv{l}'_2} \right> + \cdots \right]^\ddagger \nonumber \\
	&\;\;\;\; + \left[ \left< a_{\vv{l}_1} a_{\vv{l}_2} \right> \left< a_{\vv{l}'_1} a_{\vv{l}'_2} \right> \left< a_{\vv{l}_3} a_{\vv{l}'_3} \right> + \left< a_{\vv{l}_1} a_{\vv{l}_2} \right> \left< a_{\vv{l}'_2} a_{\vv{l}'_3} \right> \left< a_{\vv{l}_3} a_{\vv{l}'_1} \right> + \cdots \right]^{\ddagger\ddagger} \label{eqn:bispectrum_variance_contractions2}.
\end{align}
\eqref{eqn:bispectrum_variance_contractions2} contains all possible contractions of the 6 $a_\vv{l}$s, 3 from each of $B^{obs}_\vv{L}$ and $B^{obs}_{\vv{L}'}$. There are 15 such contractions: 6 consisting only of terms between the two bispectra (in $\ddagger$) and 9 which also have internal contractions ($\ddagger\ddagger$). Our aim is to modify the regression variables so that the resulting covariance matrix reduces to identity. The error in each $\vv{L}$ is required to be independent and have unit variance.

Terms in ($\ddagger$) originate from the symmetry present in the bispectrum. $B^{obs}_\vv{L} = a_{\vv{l}_1}  a_{\vv{l}_2}  a_{\vv{l}_3}$ is invariant under permutations of ${\vv{l}_1, \vv{l}_2, \vv{l}_3}$. In order to remove duplicate elements, we restrict our $\vv{L}$s to ones satisfying $\vv{l}_1 \le \vv{l}_2 \le \vv{l}_3$. \footnote{$(l_1,m_1) < (l_2,m_2)$ if and only if ($l_1 < l_2$) \textbf{or} ($l_1 = l_2$ and $m_1 < m_2$).}

On the other hand, terms in ($\ddagger\ddagger$) reveal a more complex issue about our formulation. Observed anisotropies $a_\vv{l}$ are not necessarily independent, as various factors such as partial sky coverage and correlated noise can induce correlations between them. Even if they are independent, our construction of $B^{obs}_\vv{L}$ can yield non-zero terms in ($\ddagger\ddagger$) when at least two out of $\vv{l}_1, \vv{l}_2, \vv{l}_3$ are identical. This is similar in essence to incorrectly estimating variance of a random variable $X$ from samples $X_i$ by computing $\sum_i X_i^2$, instead of $\sum_i (X_i- \left< X \right>)^2$.

We redefine the observed bispectra in terms of $a_\vv{l}$s by subtracting off the extra contributions.
\begin{align}
	{B'}^{obs}_\vv{L} := a_{\vv{l}_1} a_{\vv{l}_2} a_{\vv{l}_3} - \left< a_{\vv{l}_1} a_{\vv{l}_2} \right> a_{\vv{l}_3} - \left< a_{\vv{l}_2} a_{\vv{l}_3} \right> a_{\vv{l}_1} - \left< a_{\vv{l}_3} a_{\vv{l}_1} \right> a_{\vv{l}_2}. \label{def:bispectrum_estimate_including_linear}
\end{align}
The newly introduced terms linear in $a_\vv{l}$s have mean contribution of zero, since $\left< a_\vv{l} \right> = 0$ except for the monopole $l=0$ which is excluded from CMB bispectrum analysis. Thus ${B'}^{obs}_\vv{L}$ is still an unbiased estimate of the underlying bispectra. Substituting into \eqref{eqn:bispectrum_variance_contractions1}, all terms previously in ($\ddagger\ddagger$) now vanish.

In theory, we can compute \eqref{eqn:bispectrum_variance_contractions1} using the full covariance matrix $C_{\vv{l}_1 \vv{l}_2} = \left< a_{\vv{l}_1} a_{\vv{l}_2} \right>$ and invert it to get the least squares estimate for $f_{NL}$s. In practice, however, this is a costly and numerically demanding operation. We approximate the non-diagonal covariances in \eqref{def:bispectrum_estimate_including_linear} using Monte Carlo method: $\langle a_{\vv{l}_1}^{G} a_{\vv{l}_2}^{G} \rangle_{MC}$ from ensemble of realistic Gaussian simulations. Otherwise, we assume $C_{\vv{l}_1 \vv{l}_2} \approx C_{l_1} \delta_{\vv{l}_1 \vv{l}_2}$ to simplify our calculations.

Each element of the regressand is now independent;
\begin{align}
	\left< {B'}^{obs}_\vv{L} {B'}^{obs}_{\vv{L}'} \right> = C_{l_1} C_{l_2} C_{l_3} \; \Delta_{\vv{l}_1 \vv{l}_2 \vv{l}_3} \; \delta_{\vv{l}_1 \vv{l}'_1} \delta_{\vv{l}_2 \vv{l}'_2} \delta_{\vv{l}_3 \vv{l}'_3},
\end{align}
where $\Delta_{\vv{l}_1 \vv{l}_2 \vv{l}_3}$ is a symmetry factor equal to $6$ if $\vv{l}_1 = \vv{l}_2 = \vv{l}_3$, $2$ if exactly two of them are identical, and $1$ if all are distinct. The regressand and regressors are rescaled so that each element of the error term has unit variance.
\begin{align}
	\tilde{B}^{obs}_\vv{L} &:= \frac{1}{\sqrt{\Delta_{\vv{l}_1 \vv{l}_2 \vv{l}_3} C_{l_1} C_{l_2} C_{l_3}}} \left[ a_{\vv{l}_1} a_{\vv{l}_2} a_{\vv{l}_3} - \left< a_{\vv{l}_1} a_{\vv{l}_2} \right> a_{\vv{l}_3} - \left< a_{\vv{l}_2} a_{\vv{l}_3} \right> a_{\vv{l}_1} - \left< a_{\vv{l}_3} a_{\vv{l}_1} \right> a_{\vv{l}_2} \right] \\
	\tilde{B}^{(i)}_\vv{L} &:= \frac{1}{\sqrt{\Delta_{\vv{l}_1 \vv{l}_2 \vv{l}_3} C_{l_1} C_{l_2} C_{l_3}}} \; \mathcal{G}_{\vv{L}} b^{(i)}_{L}
\end{align}

The ordinary least squares estimate for a linear model $\vv{y} = X\vv{\beta} + \vv{\epsilon}$ is given by $\hat{\beta} = (X^T X)^{-1} X^T \vv{y}$. We define the equivalent objects to $(X^T X)_{ij}$ and $(X^T \vv{y})_i$ in our linear model \eqref{eqn:bispectrum_estimation_core} as $F_{ij}$ and $S_i$, respectively. The matrix $F$ is given by
\begin{align}
	F_{ij} &:= \tilde{\vv{B}}^{(i)} \cdot \tilde{\vv{B}}^{(j)} \\[0.5ex]
	&= \sum_{\vv{l}_1 \le \vv{l}_2 \le \vv{l}_3}  \frac{\mathcal{G}_{\vv{L}}^2 \; b^{(i)}_{L} b^{(j)}_{L}}{\Delta_{\vv{l}_1 \vv{l}_2 \vv{l}_3} \; C_{l_1} C_{l_2} C_{l_3}}   
	= \sum_{\vv{l}_1 , \vv{l}_2 , \vv{l}_3}  \frac{\mathcal{G}_{\vv{L}}^2 \; b^{(i)}_{L} b^{(j)}_{L}}{6 \; C_{l_1} C_{l_2} C_{l_3}}
	= \sum_{l_1,l_2,l_3}  \frac{h^2_L \; b^{(i)}_{L} b^{(j)}_{L}}{6 \; C_{l_1} C_{l_2} C_{l_3}}. \label{eqn:bispectrum_estimation_fisher}
\end{align}
Note that in the symmetry factor in \eqref{eqn:bispectrum_estimation_fisher} was removed by reintroducing the sum over all possible $\vv{l}_j$s. For the last equality we used $h^2_{l_1 l_2 l_3}$ defined in \eqref{def:h2_using_gaunt_integral}.

Similarly, the vector $S$ can be written as
\begin{align}
	S_i &:= \tilde{\vv{B}}^{(i)} \cdot \tilde{\vv{B}}^{obs} \\
	&= \sum_{\vv{l}_1 , \vv{l}_2 , \vv{l}_3} \frac{\mathcal{G}_{\vv{L}} b^{(i)}_{L}}{6 \; C_{l_1} C_{l_2} C_{l_3}} \left[ a_{\vv{l}_1} a_{\vv{l}_2} a_{\vv{l}_3} - \left< a_{\vv{l}_1} a_{\vv{l}_2} \right> a_{\vv{l}_3} - \left< a_{\vv{l}_2} a_{\vv{l}_3} \right> a_{\vv{l}_1} - \left< a_{\vv{l}_3} a_{\vv{l}_1} \right> a_{\vv{l}_2} \right].
	\label{eqn:bispectrum_estimation_signal}
\end{align}
Finally, we have the least squares estimate of $f_{NL}$ in terms of $F$ and $S$;
\begin{align}
	\hat{f}_{NL}^{(i)} = \sum_j (F^{-1})_{ij} S_j. \label{eqn:bispectrum_estimation_ols_estimate}.
\end{align}

One of the main strengths of the ordinary least squares estimator lies in its \textit{optimality}; it has the smallest possible variance, named the Cramer-Rao bound, among all unbiased estimators that are linear in data. Noting that $\left< S_i S_j \right> = F_{ij}$, we get $\text{Var}(\hat{f}_{NL}^{(i)}) = (F^{-1})_{ii}$. This value is indeed equal to the Cramer-Rao bound computed from the Fisher information matrix, which is $F$ here.

Fisher matrix of the estimator naturally motivates the following definition of inner product on the space of reduced bispectra.
\begin{align}
	\left< \vv{b}^{(i)}, \vv{b}^{(j)} \right> &:= \sum_L \frac{h^2_L b^{(i)}_L b^{(j)}_L}{6 \; C_{l_1} C_{l_2} C_{l_3}}  = F_{ij}.	
\end{align}
In particular, correlation between two bispectrum shapes can be defined as
\begin{align}
	\text{Corr} \left( \vv{b}^{(i)}, \vv{b}^{(j)} \right) &:= \frac{ \left< \vv{b}^{(i)}, \vv{b}^{(j)} \right>}{\sqrt{ \left< \vv{b}^{(i)}, \vv{b}^{(i)} \right> \left< \vv{b}^{(j)} \vv{b}^{(j)} \right> }}.
\end{align}
If all bispectrum shapes under consideration are uncorrelated in this metric so that $| \text{Corr}(\vv{b}^{(i)}, \vv{b}^{(j)}) | \ll 1$ whenever $i \neq j$, then the Fisher matrix $F$ is approximately diagonal, and $(F^{-1})_{ii} \approx (F_{ii})^{-1}$. In this case, estimated $\hat{f}_{NL}^{(i)}$ in \eqref{eqn:bispectrum_estimation_ols_estimate} is identical to the value obtained using a single regressor $\vv{B}^{(i)}$. In other words, we may analyse individual bispectrum shapes independently.


\subsection{CMB bispectrum estimators}

Review of KSW, Modal and Binned estimators.


\subsection{Lensing-ISW bias}

Review lensing-ISW bias