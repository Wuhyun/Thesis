%!TEX root = ../thesis.tex
%*******************************************************************************
%****************************** Sixth Chapter **********************************
%*******************************************************************************
\chapter{Conclusion}

The standard model of cosmology has been remarkably successful; the CMB power spectrum from Planck, for example, shows an exquisite fit to the $\Lambda$CDM model with only six free parameters. There are many unanswered questions remaining, however, especially regarding the early universe physics. The most widely accepted theory for the early universe is inflation, whose prediction of the scale-invariant primordial spectra has been tested to be consistent with the observed data. Inflation also resolves the horizon and flatness problems present in the vanilla Big Bang cosmology. The simplest scenario of the single-field slow-roll inflation with a canonical kinetic term can explain the current observations, but there are also numerous other physically well-motivated scenarios that are yet to be ruled out.

The key to distinguishing different models of inflation is primordial non-Gaussianity. Deviations from the most canonical inflation model leave weak non-Gaussian signatures in the primordial perturbations. These imprints are best captured in the bispectrum of the primordial perturbations. Studying the shape and amplitude of the primordial bispectrum therefore allows us to constrain various inflation models directly. The CMB is the ideal probe for this job since its anisotropy depends linearly on the initial density perturbations. The CMB bispectrum hence directly relates to the primordial bispectrum and is used to construct the optimal estimator for the primordial non-Gaussianity parameter $f_\text{NL}$. 

The most recent Planck analysis has constrained $f_\text{NL}$ to great precision using the CMB bispectrum estimator. No primordial source of non-Gaussianity has been found, and $f_\text{NL}$s for the standard shapes are currently consistent with $0$. More interesting `hints' of non-Gaussianity have been found in models with oscillations in the bispectrum. Signals of $3$-$4\sigma$ significance have been found in these models, but scanning over a large parameter space meant that these are not yet conclusive detections. A further investigation using independent and more precise measurements would be extremely beneficial.

The next generation of CMB experiments is anticipated to measure the polarisation of the CMB with greatly enhanced sensitivity. Given that the Planck constraints on oscillatory models benefited much more from the inclusion of polarisation data than other shapes, we predict that the constraining power would benefit immensely from the future CMB data. Our work presented in Chapter 4 forecasts that the most sensitive CMB Stage-4 experiment specification is expected to yield a factor of 1.7-2.2 times more stringent constraints compared to Planck.

Despite the bright prospects and growing interest in constraining oscillatory models, a large part of the model and parameter space are currently unconstrained due to numerical and computational difficulties. The CMB bispectrum estimation is a challenging task where the na\"ive computation is practically impossible. There are two main approaches to this: KSW and Modal. The KSW estimator exploits the inherent separability of the primordial bispectrum template to efficiently compute the multi-dimensional integrals involved. This method can accurately constrain even highly oscillatory shapes but is severely restrictive in the type of bispectrum shapes it can handle. On the other hand, the Modal estimator expands the primordial and late-time bispectra in terms of separable mode functions to tackle the problem. The Modal code is heavily optimised and can constrain a wide range of non-separable shapes. However, the Modal estimator has trouble dealing with high-frequency oscillations, since the basis expansion struggles to converge. Choosing a specialised basis set to handle oscillations is also complicated by the fact that there are two separate basis sets\textemdash primordial and late-time\textemdash to consider.

Our novel approach to the CMB bispectrum estimation, \textsc{CMB-BEst}, is designed to combine the strengths of the KSW and Modal estimators, and hence is suitable for constraining general oscillatory models using future CMB data. We use the primordial basis expansion to decompose the given bispectrum shape into separable terms and then apply a KSW-like method to obtain constraints on each of them. \textsc{CMB-BEst} works for general basis sets and is in fact equivalent to the KSW estimator for a simple choice of basis. It can handle high-frequency oscillations as accurately as the KSW estimator and applies to general non-separable shapes like the Modal estimator.

There is no free lunch, however. The \textsc{CMB-BEst} pipeline is computationally more costly than both the KSW and Modal ones. We therefore invested a significant amount of time optimising the code at both algorithmic and implementation levels. The code was then parallelised on multiple levels to fully benefit from the modern computing architecture. The total runtime has improved many orders of magnitude over the course of this process. The completed code has then thoroughly tested against the Modal pipeline used in the Planck 2018 analysis. We found that the results are consistent with Planck map-by-map using 140 simulated maps, for the standard shapes as well as the feature and resonance templates.

\textsc{CMB-BEst} has been validated and is now in the exploitation phase. Working in collaboration with the authors of \textsc{Primodal}, we developed a fluid pipeline where a given inflationary Lagrangian can be directly constrained without the use of approximate templates. This allows a fast and accurate constraint to the specific model under consideration. As a proof-of-concept example, we presented work on constraining the DBI sound speed in Chapter 5, where we obtained $c_\text{s}^\text{DBI} \ge 0.056$ with 95\% confidence. The combined \textsc{Primodal}+\textsc{CMB-BEst} pipeline can perform similar scans on other theoretical parameters, which we plan to do in the near future.

There are currently two main routes for improvement for \textsc{CMB-BEst}. First, the code can be generalised to incorporate the E-mode polarisation data. Implementing this should be straightforward using an orthonormalisation defined in Chapter 4. The additional computational complexity involved in the extra set of spherical harmonic transforms for polarisation is expected to be subdominant and have a small impact overall. Second, we plan on improving the current method used for the decomposition of a given primordial shape function. We will investigate further the effect of the integration domain, as we saw from the example of the DBI resonance models in Chapter 5.

It is truly an exciting time to be researching cosmology. Numerous future experiments will soon provide us with ever more immense and accurate measurements of the observable universe. We believe that the work presented here will add to the community's continued efforts to better understand the universe we live in.