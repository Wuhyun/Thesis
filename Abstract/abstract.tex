% ************************** Thesis Abstract *****************************
% Use `abstract' as an option in the document class to print only the titlepage and the abstract.
\begin{abstract}

The Cosmic Microwave Background (CMB) is one of the most valuable probes of the universe we have today. Anisotropies present in the ancient light contain rich statistical information about the perturbations in the early universe and their subsequent evolution until now. CMB bispectrum, the Fourier equivalent of three-point correlation function, allows us to study weak non-Gaussian signatures of the primordial fluctuations. Primordial non-Gaussianity is a key prediction of many physically well-motivated inflation models, and measuring its shape and amplitude allows us to constrain them.

This thesis comprises two sections. In the first section, we present forecasts on primordial non-Gaussianity constraints from upcoming CMB Stage-4 surveys. We focus our attention to models favoured by the Planck analysis, where a sharp feature in either the inflationary potential or sound speed causes oscillations in the bispectrum. Using preliminary specifications, we find that the Simons Observatory will have up to a factor of 1.6 improvements over Planck, increased to 1.7-2.2 for CMB Stage-4 experiments. Motivated by bright prospects, we developed a novel CMB bispectrum estimator suited for the resolution and sensitivity of future surveys.

We discuss our high-resolution bispectrum estimator in the second section. Our code, named CMB-BEst, utilises a set of general basis functions to accurately constrain a wide variety of models. Implementing such flexible and precise estimator was a computationally challenging task. We detail our algorithm design, code optimisation and parallelisation for high performance computing clusters, which made the computation tractable at all. Validation tests on internal consistency and comparison against a conventional estimator are provided, together with a proof-of-concept application. We highlight how CMB-BEst can be used for both general and targeted analyses of previously unconstrained models.

\end{abstract}
