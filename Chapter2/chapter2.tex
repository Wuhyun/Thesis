%!TEX root = ../thesis.tex
%*******************************************************************************
%****************************** Second Chapter *********************************
%*******************************************************************************

\chapter{Cosmic Microwave Background Anisotropy}

\ifpdf
    \graphicspath{{Chapter2/Figs/Raster/}{Chapter2/Figs/PDF/}{Chapter2/Figs/}}
\else
    \graphicspath{{Chapter2/Figs/Vector/}{Chapter2/Figs/}}
\fi

Cosmic Microwave Background is like a trove full of precious cosmological information.
(TODO: discovery led to confirmation of big bang. Anisotropy constrains cosmological parameters, almost by itself. success of LCDM. Bispectrum. Lensing map and secondaries like tSZ, kSZ...)



\begin{itemize}
	\item Explain recombination and CMB, without talking about detailed thermal history 
	\item CMB perfectly follows blackbody spectrum. Borrow a figure of intensity vs wavelength from Planck
	\item Existence of temperature anisotropy. Maps from COBE, WMAP and Planck
	\item Explain goal: obtain theoretical derivations for CMB anisotropy and introduce statistical formulation for comparing with measurements.	
\end{itemize}

\section{The inhomogeneous universe}

\subsection{Metric perturbations} \label{section:metric_perturbations}
Recall that the FLRW metric is given using conformal time by
\begin{align}
	ds^2 = \bar{a}(t)^2 (-d\tau^2 + d\vv{x}^2),
\end{align}
where a bar is used to denote quantities computed from the homogeneous universe. We write perturbations around this background metric as follows.
\begin{align}
	ds^2 = \bar{a}(t)^2 \left[ -(1+2A) dt^2 + 2B_i \; d\tau dx^i + (\delta_{ij} + h_{ij}) dx^i dx^j \right].
\end{align}
The spatial indices $i,j,\cdots$ are lowered and raised using $\delta_{ij}$ here. Note that the scale factor has not been perturbed, since any variation of it can be absorbed into other perturbative variables. There are 1, 3 and 6 degrees of freedom coming from $A$, $B_i$ and $h_{ij}$, respectively, adding up to 10 as expected from a 3+1 dimensional spacetime.

We further extract the divergence part from $B_i$ and $h_{ij}$, as well as the trace of $h_{ij}$. Writing $V$ and $T$ to denote vector and tensor quantities, 
\begin{align}
	B_i &= \partial_i B + B_i^V \\
	h_{ij} &= 2C \delta_{ij} + 2\partial_{\langle i}\partial_{j \rangle} E + (\partial_i E_j^V + \partial_j E_i^V). + 2E^T_{ij},
\end{align}
where $\partial_{\langle i}\partial_{j \rangle} := \partial_i \partial_j - \frac{1}{3} \delta_{ij} \nabla^2$.
The variables are chosen such that $B_i^V$ has vanishing divergence, and $h_{ij}^T$ is traceless and transverse.

The Scalar-Vector-Tensor (SVT) theorem states that to linear order in the perturbations, these modes decouple and evolve in three independent groups: scalars ($A,B,C,E$), vectors ($B_i^V,E_i^V$), and tensors ($E_{ij}^T$). Each of these contain 4, 4, and 2 degrees of freedom, respectively, again totalling to 10 as required. In this section, we are only interested in scalar perturbations which generates temperature anisotropy and E-mode polarisation in the CMB.

Keeping only the scalar modes, the perturbed metric becomes
\begin{align}
	ds^2 = \bar{a}(\tau)^2 \left[ -(1+2A) dt^2 + 2 \partial_i B \; d\tau dx^i + \left( (1+2C)\delta_{ij} + 2\partial_{\langle i}\partial_{j \rangle} E \right) dx^i dx^j \right]. \label{eqn:perturbed_metric_scalar}
\end{align}

Coordinate transformation are \textit{gauge} symmetries in General Relativity; they correspond to redundancies in our mathematical representation of the system. Redefining the coordinate variables may change how the theory looks like, but all physical results derived the new set are equivalent to the original theory and are just a coordinate transformation away.

Consider gauge transformations $x^\mu \rightarrow \tilde{x}^\mu = x^\mu + \xi^\mu$ for some small $\xi$. As long as $\xi$ is of the same order as other perturbative variables, the background metric remains to be FLRW. Scalar perturbations are affected by transformations generated from two arbitrary functions: $\xi^0 = T(\tau,\vv{x})$ and $\xi^i = \partial_i L(\tau,\vv{x})$. The transformation matrices take the form
\begin{align}
	\frac{d\tilde{x}^\alpha}{dx^\mu} = \begin{pmatrix}
		1 + T' & \partial_i T \\ \partial_i L' & \delta_{ij} + \partial_i \partial_j L
	\end{pmatrix},
	\hspace{0.05\linewidth}
	\frac{dx^\mu}{d\tilde{x}^\alpha} = \begin{pmatrix}
		1 - T' & -\partial_i T \\ -\partial_i L' & \delta_{ij} - \partial_i \partial_j L
	\end{pmatrix}.
\end{align}
Metric tensor in the new set of coordinates can be found using the tensor transformation rule
\begin{align}
	g_{\mu\nu} \rightarrow \tilde{g}_{\alpha\beta} = \frac{dx^\mu}{d\tilde{x}^\alpha} \frac{dx^\nu}{d\tilde{x}^\beta} g_{\mu\nu}.
\end{align}
Note also that the scale factor also changes 
metric transforms due to variations in its argument;  $\bar{a}(\tau)^2 \rightarrow \bar{a}(\tilde{\tau})^2 = \bar{a}(t)^2 (1 + 2 \mathcal{H} T)$. After some calculations we see that each perturbation variables in the metric \eqref{eqn:perturbed_metric_scalar} becomes
\begin{alignat}{2}
	\tilde{A} &= A - T' - \mathcal{H}T, \qquad &&\tilde{B} = B + T - L', \label{eqn:gauge_transform_perturbations_1}\\
	\tilde{C} &= C - \mathcal{H}T - \frac{1}{3}\nabla^2 L, \qquad &&\tilde{E} = E - L. \label{eqn:gauge_transform_perturbations_2}
\end{alignat}
Hence, the two free functions $T$ and $E$ can be chosen in a way that the metric perturbations satisfy some desirable properties. Doing so \textit{fixes} the gauge and no further redundancies remain in our formulation. Popular choices include the \textit{spatially flat} gauge, where $C=E=0$ and spatial perturbations vanish as $h_{ij}=0$, and the \textit{synchronous} gauge, for which $A=B=0$, leaving time unperturbed.

Meanwhile, it is possible to construct quantities from perturbative variables such that they remain invariant under the transformations \eqref{eqn:gauge_transform_perturbations_1}-\eqref{eqn:gauge_transform_perturbations_2}. The Bardeen potentials are two such examples;
\begin{align}
	\Psi_B &:= A + \mathcal{H}(B-E') + B' - E'', \\
	\Phi_B &:= -C - \mathcal{H}(B-E') + \frac{1}{3}\nabla^2 E.
\end{align}
The gauge we will be using for the rest of this chapter is \textit{Newtonian} gauge with $B=E=0$. The metric is then rewritten in terms of the Bardeen potentials (dropping the subscript):
\begin{align}
	ds^2 = a(\tau)^2 \left[ -(1 + 2\Psi) d\tau^2 + (1-2\Phi) d\vv{x}^2  \right]. \label{eqn:perturbed_metric_newtonian_gauge}
\end{align}
Perturbations $\Psi(\tau,\vv{x})$ and $\Phi(\tau,\vv{x})$ are equivalent to the classical gravitational potential in the Newtonian limit.


\subsection{Matter perturbations}

In order to complete the Einstein field equation, we need the energy-momentum tensor as well as the metric written to first order in perturbations. Recall that a perfect fluid in homogeneous universe has $\bar{T}_{\mu\nu} = (\bar{\rho} + \bar{P}) \bar{U}_\mu \bar{U}_\nu + \bar{P} \bar{g}_{\mu\nu}$. Not only the perturbations in energy density and pressure, but also variations in the metric and comoving observer's four-velocity affects $T_{\mu\nu}$. We consider the local inertial frame to separate these two effects. Let
\begin{align}
	(E_0)^\mu := a^{-1} (1 - \Psi) \delta^\mu_0, \quad (E_i)^\mu := a^{-1} (1 + \Phi) \delta^\mu_i. \label{eqn:newtonian_gauge_tetrad}
\end{align}
The four vectors form an orthonormal frame where $g_{\mu\nu}(E_\alpha)^\mu (E_\beta)^\nu = \eta_{\alpha\beta}$. In this locally Minkowski frame, each component of the tensor $\tilde{T}^{\alpha\beta}$ corresponds to physical quantities. We write
\begin{align}
	&\tilde{T}^{00} = \rho = \bar{\rho} + \delta\rho, \qquad \tilde{T}^{0i} = q^i = (\bar{\rho}+\bar{P})v^i, \\
	&\tilde{T}^{ij} = P \delta^{ij} - \Pi^{ij} = (\bar{P} + \delta P) \delta^{ij} - \Pi^{ij}.
\end{align}
The momentum density $\vv{q}$ and mean velocity $\vv{v}$ are first-order quantities which vanish in the homogeneous universe. Symmetric and traceless matrix $\Pi^{ij}$ represent anisotropic stress. This quantity vanishes for non-relativistic fluid such as dark matter, while it remains small but non-zero for relativistic particles including photons.

Transforming back to the original coordinates by $T^{\mu\nu}=(E_\alpha)^\mu (E_\beta)^\nu \tilde{T}^{\alpha\beta}$,
\begin{alignat}{2}
	T^{00} &= a^{-2} \bar{\rho} &&\;\;+ a^{-2}\left( \delta\rho - 2\Psi\bar{\rho} \right) \label{eqn:perturbed_energy_momentum_tensor_1}\\
	T^{0i} &= 0 &&\;\;+ a^{-2} q^i \label{eqn:perturbed_energy_momentum_tensor_2}\\
	T^{ij} &= a^{-2} \bar{P} \delta^{ij} &&\;\;+ a^{-2} \left[ (\delta P + 2\Phi\bar{P})\delta^{ij} - \Pi^{ij} \right]\label{eqn:perturbed_energy_momentum_tensor_3}
\end{alignat}
to first order in perturbations. Note that when only scalar modes are considered, we may write $v^i = \partial_i v$ and $\Pi^{ij} = \partial_{\langle i} \partial_{j \rangle} \Pi$ for some $v$, $\Pi$. If multiple fluids contribute, then the total energy-momentum tensor is obtained from summing over all individual values; $\delta\rho = \sum_I \delta\rho_I$, $\delta P = \sum_I \delta P_I$, and $\delta \vv{q} = \sum_I \vv{q}_I$.


\subsection{Initial perturbations}

With metric and matter perturbations written down, we only need one more ingredient to be able to solve linearised Einstein's equations: initial conditions. In inflationary $\Lambda CDM$, modulations about the background solution originate from quantum fluctuations of the inflation field as we discussed in Section \ref{section:quantum_fluctuations}. Statistical properties of the perturbations depend on the inflationary scenario.

In models where a single field drives inflation, such as slow-roll inflation, the field can be thought of as a local \textit{clock}. Classical equations of motions dictate the trajectory followed by the field which regulates the expansion of the universe. Any spatial fluctuations of the inflation field can therefore be considered as differences in how far it has moved down the trajectory at each point. Resulting perturbations are then determined solely by variations in the local `clock' time $\delta\tau(\tau,\vv{x})$;
\begin{align}
	\delta f \;=\; \bar{f}(\tau + \delta\tau, \vv{x}) - \bar{f}(\tau,\vv{x}) \;=\; \frac{\partial\bar{f}}{\partial\tau} \delta\tau,
\end{align}
for any $f$. By substituting $f=\rho_I$ and $P_I$ we find that $\delta\rho_I/\bar{\rho}'_I = \delta P_I / \bar{P}'_I = \delta\tau$, which is locally constant for each constituent $I$. Combined with the background continuity equation \ref{eqn:continuity},
\begin{align}
	\frac{\delta_I}{1+w_I} = \frac{\delta_J}{1+w_J}, \label{eqn:adiabatic_modes_density_contrast}
\end{align}
where we have defined the density contrast $\delta_I := \delta\rho_I / \bar{\rho}_I$. \footnote{To avoid confusion with the differential operator and Dirac delta function, we always write the density contrast $\delta_I$ with a subscript $I$ indicating which fluid it represents. The \textit{total} density contrast is denoted $\delta_{tot}$.} In particular, $\delta_r = (4/3) \delta_m$ at each point $\vv{x}$. Furthermore, for a fluid with constant equation of state $w_I = \bar{P_I} / \bar{\rho_I}$, the \textit{sound speed} $c_s$ is given by
\begin{align}
	c_s^2 := \frac{\delta P_I}{\delta\rho_I} = \frac{\bar{P}'_I}{\bar{\rho}'_I} = \frac{d\bar{P}_I}{d\bar{\rho}_I} = w_I. \label{eqn:adiabatic_modes_sound_speed}
\end{align}
Hence, the perturbed energy density and pressure also satisfy $P_I = w_I \rho_I$.

Perturbations satisfying the strong constraint of \eqref{eqn:adiabatic_modes_density_contrast} are called \textit{adiabatic} modes. They affect every component equally so that the ratio between density contrasts is uniform in space. Orthogonal to adiabatic modes are \textit{isocurvature} modes where the total density contrast vanishes everywhere; $\delta_r = -\delta_m$. Some multi-field inflation models are expected to seed isocurvature perturbations. However, all observations so far are consistent with purely adiabatic initial conditions. (TODO: add references here)

\section{Linearised equations}

In this section, we derive equations governing time evolution of the perturbations defined in the previous section. The equations are linear in the perturbative variables to first order. Although all calculations are done in Newtonian gauge, the methodology illustrated here is general and applies to any choice of gauge. For reasons that will become clear, we use $\eta$ instead of $\tau$ for conformal time in this section.

\subsection{Kinematics}

Perfect fluids which does not interact with other species, such as cold dark matter, are conserved. They should satisfy the perturbed metric's equivalent of the continuity equation \eqref{eqn:continuity}, as well as the Euler equations regarding time evolution of perturbative velocity field $\vv{v}(\eta,\vv{x})$. In this section, we derive conservation equations for the perturbation theory from the constraint $\nabla_\nu {T^\nu}_\mu = 0$ mainly for dark matter.

The first step is to compute the Christoffel symbols, or components of the Levi-Civita connection. Working in Newtonian gauge, they are given by
\begin{alignat}{6}
	&\Gamma^{0}_{00} &&= \mathcal{H} \;\;&&+ \Psi', \qquad &&\Gamma^{0}_{i0} &&= 0 \;\;&&+ \partial_i\Psi', \label{eqn:perturbed_christoffel_symbols_1}\\
	&\Gamma^{0}_{ij} &&= \mathcal{H}\delta_{ij} \;\;&&- \left[\Phi' + 2\mathcal{H}(\Psi+\Phi) \right]\delta_{ij}, \qquad &&\Gamma^{i}_{00} &&= 0 \;\;&&+ \partial_i\Psi, \label{eqn:perturbed_christoffel_symbols_2}\\
	&\Gamma^{i}_{jk} &&= 0 \;\;&&+ (\partial_k \Phi)\delta_{ij} - (\partial_i \Phi)\delta_{jk} - (\partial_j \Phi)\delta_{ik}, \qquad &&\Gamma^{i}_{j0} &&= \mathcal{H}\delta_{ij} \;\;&&- \Phi'\delta_{ij}. \label{eqn:perturbed_christoffel_symbols_3}
\end{alignat}
The background values at zeroth order in perturbations appear as the first term in each equation. Note that these are different to \eqref{eqn:homogenous_christoffel_1}-\eqref{eqn:homogenous_christoffel_3} because the conformal time $\eta$ is used here instead of comoving time $t$. Spatial indices are lowered and raised using delta function in above.

There are four components in equation $\nabla_\nu {T^\nu}_\mu = \partial_\nu {T^\nu}_\mu + \Gamma^\nu_{\nu\gamma} {T^\gamma}_\mu - \Gamma^\gamma_{\nu\mu} {T^\nu}_\gamma=0$. We take $\mu=0$ and substitute in the perturbed energy-momentum tensor \eqref{eqn:perturbed_energy_momentum_tensor_1}-\eqref{eqn:perturbed_energy_momentum_tensor_3}. After removing the background part $\bar{\rho}'=-3\mathcal{H}(\bar{\rho}+\bar{P})$, we obtain
\begin{align}
	(\delta\rho)' + \nabla \cdot \vv{q}+ 3 \mathcal{H} (\delta\rho + \delta P) - 3 \Phi' (\bar{\rho} + \bar{P})  = 0.  \label{eqn:perturbed_continuity_equation}
\end{align}
This is the continuity equation. Only the first two terms would remain if the spacetime is flat; change in energy density directly relates to fluid flow. The third term accounts for dilution of fluid density caused by expansion of the universe. The last term correct for perturbations in the expansion, as can be seen form the effective spatial scale factor $a(\eta)(1-\Phi(\eta,\vv{x}))$. All factors of three come from having three spatial dimensions.

\eqref{eqn:perturbed_continuity_equation} applies to every fluid component which does not interact with others. Rewriting in terms of the equation of state $w_I = \bar{P}_I / \bar{\rho}_I$, sound speed $c_s^2 = \delta P / \delta\rho$, and density contrast $\delta_I = \delta\rho_I / \bar{\rho}_I$, we have
\begin{align}
	\delta'_I + (1+w_I) \nabla\cdot\vv{v}_I + 3\mathcal{H}(c_s^2-w_I)\delta_I - 3(1+w_I) \Phi' = 0.
\end{align}

Meanwhile, the $\mu=i$ part of $\nabla_\nu {T^\nu}_\mu =0$ yields the Euler equations;
\begin{align}
	\vv{v}'_I + \mathcal{H} \left( 1 - \frac{3\bar{P}'}{\bar{\rho}'} \right) \vv{v}_I + \frac{\nabla(\delta P)}{\bar{\rho} + \bar{P}} + \nabla\Psi = 0. \label{eqn:perturbed_Euler_equation}
\end{align}
We do not have zeroth order terms for this case since $\vv{v}=\vv{0}$ in the homogeneous universe. Although \eqref{eqn:perturbed_Euler_equation} is a vector equation, it has only one independent scalar component after rewriting $\vv{v}=\nabla v$.

Physical meaning of individual terms in \eqref{eqn:perturbed_Euler_equation} is as follows. The second term with $\mathcal{H}$ represent dilution, or redshift, due to the expansion. The third is from pressure; the fluid's velocity accelerates in the direction of steepest pressure gradient. Lastly, the gravitational force pulling towards a potential well is encapsulated in the final term.

If we assume adiabatic initial conditions where $\bar{P}'/\bar{\rho}' = \delta P / \delta \rho = c^2_s$, then the Euler equation further simplifies to
\begin{align}
	\vv{v}' + \mathcal{H} (1 - 3c^2_s) \vv{v} + \frac{c^2_s}{1+w_I} \nabla\delta_I + \nabla\Psi = 0,
\end{align}
for a given fluid component $I$.


\subsection{Dynamics}

Our next step is to solve linearised Einstein's equations. Using the results for Christoffel symbols \eqref{eqn:perturbed_christoffel_symbols_1}-\eqref{eqn:perturbed_christoffel_symbols_3}, we may derive the Einstein tensor for the perturbed metric. After a rather long but straightforward algebra,
\begin{alignat}{2}
	G_{00} &= 3\mathcal{H}^2 \quad&&+ 2\nabla^2 \Phi - 6 \mathcal{H} \Phi', \label{eqn:perturbed_einstein_tensor_1}\\
	G_{0i} &= 0 \quad&&+ 2\partial_i \Phi' + 2 \mathcal{H} \partial_i \Psi, \label{eqn:perturbed_einstein_tensor_2}\\
	G_{ij} &= -(2\mathcal{H}' + \mathcal{H}^2) \delta_{ij} \quad&&- \partial_{\langle i} \partial_{j \rangle} (\Psi - \Phi) + \left[ \frac{2}{3}\nabla^2 (\Psi-\Phi) + 2\Phi'' \right. \nonumber\\
	& &&\quad + (4\mathcal{H}' + 2\mathcal{H}^2)(\Psi+\Phi) + 2\mathcal{H}\Psi' + 4\mathcal{H}\Phi' \biggr] \delta_{ij}, \label{eqn:perturbed_einstein_tensor_3}
\end{alignat}
where we have separated the zeroth and first order terms as before.

The Einstein's equation $G_{\mu\nu} = 8\pi G T_{\mu\nu}$ consists of 10 parts. Among these only 4 of them relate to scalar modes, similarly to how there are 4 independent scalar perturbations in the metric tensor \eqref{eqn:perturbed_metric_scalar}. We obtain one equation for each of $G_00$, $G_{i0}$, trace of $G_{ij}$, and trace-free part of $G_{ij}$.

We start with the trace-free mode of $G_{ij}$. Combining  \eqref{eqn:perturbed_einstein_tensor_3} and \eqref{eqn:perturbed_energy_momentum_tensor_3} gives
\begin{align}
	-\partial_{\langle i} \partial_{j \rangle} (\Psi - \Phi) = \partial_{\langle i} \partial_{j \rangle} \Pi.
\end{align}
The anisotropic stress $\Pi$ is negligible in reality. Non-relativistic matter does not contribute at all to $\Pi$, and photons induce non-zero but small anisotropic stress. Hence, we set $\Pi\approx 0$ and let $\Psi = \Phi$ for the rest of our derivations.

The $00$, $i0$, and $ij$ trace parts of Einstein's equations then take the form
\begin{align}
	\nabla^2 \Phi - 3 \mathcal{H} \Phi' - 3\mathcal{H}^2 \Phi &= 4\pi G \; a^2 \delta\rho, \label{eqn:perturbed_einstein_equations_1}\\
	\Phi' + \mathcal{H} \Phi &= -4\pi G \; a^2 q \label{eqn:perturbed_einstein_equations_2}\\
	\Phi'' + 3\mathcal{H} \Phi' + (2\mathcal{H}' + \mathcal{H}^2) \Phi &= 4\pi G \; a^2 \delta P. \label{eqn:perturbed_einstein_equations_3}
\end{align}
We used the background equations to remove terms of zeroth order in perturbations. The $i0$ part generally gives a vector identity but it only has one scalar degree of freedom since $\vv{q}=(\bar{\rho}+\bar{P})\vv{v} = \nabla q$ for some $q := (\bar{\rho}+\bar{P}) v$.

Time derivatives of $\Phi$ on the left hand side can be removed by putting \eqref{eqn:perturbed_einstein_equations_1} and \eqref{eqn:perturbed_einstein_equations_2} together. The result is Poisson's equation;
\begin{align}
	\nabla^2 \Phi = 4\pi G \; a^2 (\delta\rho - 3\mathcal{H}q). \label{eqn:perturbed_poisson_eqution}
\end{align}
This very much resembles the Poisson's equation for Newtonian gravity: $\nabla^2 \phi_{newt} = 4\pi G \; \rho$. We confirm our previous claim that $\Phi$ corresponds to perturbations in the Newtonian gravitational potential. \eqref{eqn:perturbed_poisson_eqution} is best solved in Fourier space where $\nabla^2\Phi(\eta,\vv{x})$ reduces to $-k^2 \tilde{\Phi}(\eta,\vv{k})$.


\subsection{Curvature perturbations}

The Bardeen potential $\Phi_B$ is an extremely useful quantity for cosmological perturbation theory. It is not only gauge-invariant, but it also represents a physically meaningful quantity\textemdash gravitational potential\textemdash in Newtonian gauge which can be solved using the Poisson's equation \eqref{eqn:perturbed_poisson_eqution}. Gauge-invariant quantities also let us connect results from different gauge choices in a consistent manner.

Another gauge-invariant variable with crucial role in cosmology is the constant density curvature perturbation, or simply \textit{curvature perturbation}, defined as
\begin{align}
	\zeta := - C + \frac{1}{3} \nabla^2 E + \mathcal{H} \frac{\delta\rho}{\bar{\rho}'}.
\end{align}
The name originates from the fact that it closely relates to the spatial curvature $R^{(3)}$ in the uniform density gauge $B=\delta\rho=0$. In the Newtonian gauge,
\begin{align}
	\zeta = \Phi + \mathcal{H}\frac{\delta\rho}{\bar{\rho}'} = \Phi - \frac{\delta\rho}{3(\bar{\rho}+\bar{P})},
\end{align}
where we used the background continuity equation $\bar{\rho}'=-3\mathcal{H}(\bar{\rho}+\bar{P})$ for the second equality. Time derivative of $\zeta$ can then be evaluated using \eqref{eqn:perturbed_continuity_equation}:
\begin{align}
	\zeta' &= \Phi' + \left[ \frac{1}{3}\nabla\cdot\vv{v} + \mathcal{H}\frac{\delta\rho + \delta P}{\bar{\rho} + \bar{P}} - \Phi' \right] + \frac{\delta\rho (\bar{\rho}' + \bar{P}')}{3(\bar{\rho}+\bar{P})^2} \\
	&= \frac{1}{3}\nabla^2 v + \frac{\mathcal{H}}{\bar{\rho} + \bar{P}} \left( \delta P - \frac{\bar{P}'}{\bar{\rho}'} \delta\rho \right).
\end{align}
The second term vanishes for adiabatic modes sourced by single field inflation, if we also assume constant equations of state as shown in \eqref{eqn:adiabatic_modes_sound_speed}. We are then left with $\zeta' = (1/3) \nabla^2 v$.

Now consider scales much larger than the comoving Hubble radius $\mathcal{H}^{-1}$, namely the `super-horizon' scales. The Fourier space equivalent of this condition is $k \ll \mathcal{H}$. The Fourier transform of $\nabla^2 v$, $-k^2 \tilde{v}$, is thus much smaller than the typical time scale $\mathcal{H}$ in this limit. It follows that $\zeta'\approx 0$; curvature perturbations are conserved in the super-horizon scales.

For this reason, the curvature perturbations play a major role in connecting inflation with initial conditions. Recall that the comoving Hubble radius $\mathcal{H}^{-1}$ shrinks during inflation. Regions of space previously in causal contact then disconnects, letting the particle horizon later at recombination to be large enough, and resolving the horizon problem discussed earlier in Section \ref{section:the_horizon_problem}. After the end of inflation, $\mathcal{H}^{-1}$ starts growing again so that physical scales we observe today move back within the horizon. Here, since $\zeta$ remains constant at super-horizon scales, the value of $\zeta$ at horizon re-entry is equal to the one evaluated at the time the mode left the horizon. $\zeta$ serves as a bridge which links quantum fluctuations during inflation to initial perturbations after inflation.


\section{CMB anisotropy}

The linearised Einstein field equations \eqref{eqn:perturbed_einstein_equations_1}-\eqref{eqn:perturbed_einstein_equations_3} dictate the time evolution of gravitational perturbations given the total energy, momentum, and pressure of the universe's components. Contributions to these characteristics from cold dark matter can be solved using the perturbed continuity equation \eqref{eqn:perturbed_continuity_equation} and Euler equations \eqref{eqn:perturbed_Euler_equation}. This section covers how perturbations in the photon temperature evolve and translate into the observed CMB anisotropy, the main cosmological dataset in this thesis.

Studying photon perturbations involves two major complications compared to cold dark matter. First, photons directly interact with baryons through Compton scattering. Their perturbations hence stay tightly coupled until recombination, when baryons form neutral hydrogen and photons start free-streaming instead. We outline in Section \ref{section:boltzmann_equations} how Boltzmann equations are used to find the time evolution of photon perturbations while accounting for scattering. Second, unlike cold dark matter whose perturbations are characterised by its density contrast $\delta\rho_m$ and velocity field $\vv{v}_m$, we require a whole hierarchy of functions to accurately describe photons. This is because photons do not necessarily travel parallel to the wavevector when perturbed; $\vv{p} \nparallel \vv{k}$. We discuss the details in Section \ref{section:boltzmann_hierarchy}.


\subsection{Boltzmann Equations} \label{section:boltzmann_equations}
\subsubsection*{Photons}

We start by calculating the path a photon takes while free-streaming within the perturbed metric \eqref{eqn:perturbed_metric_newtonian_gauge}. The 4-momentum of a photon in local coordinates satisfies $\eta_{\alpha\beta}\tilde{P}^\alpha \tilde{P}^\beta = 0$ since photons are massless. We may hence write its components as
\begin{align}
	\tilde{P}^0 =: p, \qquad \tilde{P}^i =: p \hat{p}^i,
\end{align}
where $\hat{\vv{p}}$ is a unit vector indicating the direction of propagation. We can revert back to the perturbed metric using the tetrad in \eqref{eqn:newtonian_gauge_tetrad}; since $P^\mu = (E_\alpha)^\mu \tilde{P}^\alpha$,
\begin{align}
	P^0 = a^{-1} (1 - \Psi)p, \qquad P^i = a^{-1}(1 + \Phi) p \hat{p}^i  \label{eqn:perturbed_photon_four_momentum}
\end{align}
Recall that the 4-momentum is defined as $P^\mu = dx^\mu/\lambda$ for some affine parameter $\lambda$. Using our definition above,
\begin{align}
	\frac{dx^i}{d\eta} =  \frac{dx^i}{d\lambda} \frac{d\lambda}{d\eta} = \frac{P^i}{P^0} = (1+\Psi+\Phi)\hat{p}^i	\label{eqn:perturbed_photon_position_total_derivative}
\end{align}
to first order in perturbations. Photons seem to travel more slowly in overdense regions where the gravitational potentials $\Psi,\Phi<0$.

The geodesic equation \eqref{eqn:geodesic} take the form
\begin{align}
	\frac{dP^\mu}{d\lambda} + \Gamma^\mu_{\nu\rho} P^\nu P^\rho = P^0 \left( \frac{dP^0}{d\eta} \right) + \Gamma^0_{\nu\rho} P^\nu P^\rho = 0.  \label{eqn:geodesic_equation_four_momentum}
\end{align}
Meanwhile, differentiating \eqref{eqn:perturbed_photon_four_momentum} gives
\begin{align}
	\frac{dP^0}{d\eta} = \frac{1-\Psi}{a} \left( \frac{dp}{d\eta} - \mathcal{H}p \right) - \frac{p}{a} \left( \frac{\partial\Psi}{\partial\eta} + \frac{dx^i}{d\eta} \frac{\partial\Psi}{\partial x^i}  \right). \label{eqn:perturbed_energy_total_derivative}
\end{align}
Note the distinction between total and partial derivatives; $d/d\eta$ here is taken along the trajectory $x^\mu = x^\mu(\eta)$ of a photon. We may combine \eqref{eqn:geodesic_equation_four_momentum} and \eqref{eqn:perturbed_photon_four_momentum} to obtain an expression for the total derivative of $p$;
\begin{align}
	\frac{1}{p} \left( \frac{dp}{d\eta} \right) = - \mathcal{H} + \Phi' - \hat{p}^i \partial_i \Psi. \label{eqn:perturbed_p_total_derivative}
\end{align}
Only the first term on the right hand side remains in the zeroth order to give $p\propto a^{-1}$. We confirm that photons get redshifted as the universe expands, justifying the definition of $z$ in \eqref{def:redshift}. The other two terms quantify the effect of gravitational perturbations on photon's energy. Photons in a deepening potential well has to spend some of its energy to come out of it, hence the term $\Phi'$. On the other hand, moving into a region with deeper potential provides energy equal to $-({\vv{p}}\cdot\nabla)\Psi$. Dividing by $p$ gives the last term in \eqref{eqn:perturbed_p_total_derivative}.


\subsubsection*{Distribution function} \label{section:distribution_function}
In order to fully understand physical properties of perturbations in photons, we need to study their distribution function $f(\eta,\vv{x},\vv{p})$ which measures the number of particles in a unit phase space volume. Photons in thermal equilibrium with the homogenous universe follow Bose-Einstein distribution where   
\begin{align}
	\bar{f}(\eta, p) = \left[ \exp \left\{ \frac{p}{\bar{T}(\eta)} \right\} - 1 \right]^{-1}, \label{eqn:photon_distribution_function}
\end{align}
up to a factor and the Boltzmann constant, both of which we set to one for convenience. This is blackbody radiation with temperature $\bar{T}(\eta)$.

We define the fractional temperature anisotropy $\Theta$ through the perturbed photon distribution function as follows;
\begin{align}
	f(\eta, \vv{x}, p, \hat{\vv{p}}) = \left[ \exp \left\{ \frac{p}{\bar{T}(\eta)\left[1 + \Theta (\eta,\vv{x},\hat{\vv{p}}) \right]} \right\} - 1 \right]^{-1}. \label{eqn:perturbed_photon_distribution_function}
\end{align}
We made an implicit assumption here that $\Theta(\eta,\vv{x},\hat{\vv{p}})$ does not depend on $p$, the photon energy. This claim will be justified later as we show that the Thomson scattering leaves photon energy virtually unchanged.

The distribution function \eqref{eqn:perturbed_photon_distribution_function} can be expanded to linear order in perturbations as
\begin{align}
	f = \bar{f} - \frac{\partial \bar{f}}{\partial (\ln p)} \; \Theta,  \label{eqn:perturbed_photon_distribution_function_expansion}
\end{align}
by replacing $p$ with $p(1-\Theta)$ in \eqref{eqn:photon_distribution_function}. The temperature anisotropy $\Theta$ therefore closely relates to the perturbed distribution function.

\subsubsection*{Collisionless equations} \label{section:collisionless_equation}

Liouville's theorem states that the phase space distribution function remains constant along the system's trajectories. A generalisation of this to systems with collisions is the Boltzmann equation. For our photon distribution function, it reads
\begin{align}
	\frac{df}{d\eta} &= \frac{\partial f}{\partial \eta} + \frac{\partial f}{\partial x^i}\frac{dx^i}{d\eta} + \frac{\partial f}{\partial(\ln  p)}\frac{d(\ln p)}{d\eta} + \frac{\partial f}{\partial \hat{p}^i}\frac{d\hat{p}^i}{d\eta} \label{eqn:boltzmann_equation_base}\\
	&= \left. \frac{df}{d\eta} \right|_\text{scattering} . \label{eqn:boltzmann_equation_base_scattering}
\end{align}
In \eqref{eqn:boltzmann_equation_base} we used chain rule to expand out the total derivative again. Note that the last term only appears at the second order in perturbations since both $(\partial f/\partial \hat{p}^i)$ and $(d\hat{p}^i/d\eta)$ appear at the first order. This term corresponding to gravitational lensing opens up a whole subject of its own but will be neglected for our purposes in this thesis.

Without scattering, the term in \eqref{eqn:boltzmann_equation_base_scattering} vanishes and the distribution function is conserved on each trajectory. The zeroth order part of the equation yields
\begin{align}
	\frac{d\bar{f}}{d\eta} = \frac{\partial \bar{f}}{\partial\eta} - \frac{\partial \bar{f}}{\partial(\ln p)}\frac{d(\ln\bar{p})}{d\eta} = 0. \label{eqn:boltzmann_equation_collisionless_zeroth}
\end{align}
Substituting the form of $\bar{f}$ we see that the background temperature scales as $\bar{T}\propto a^{-1}$, consistent with the redshift caused by expansion of the universe.

We can write down the first order part of \eqref{eqn:boltzmann_equation_base} in terms of the temperature anisotropy using our previous results \eqref{eqn:perturbed_photon_position_total_derivative}, \eqref{eqn:perturbed_p_total_derivative}, and \eqref{eqn:perturbed_photon_distribution_function_expansion};
\begin{align}
	\frac{df}{d\eta} &= - \frac{\partial}{\partial\eta} \left( \frac{\partial \bar{f}}{\partial(\ln p)} \Theta \right) - \frac{\partial \bar{f}}{\partial (\ln p)} \hat{p}^i \frac{\partial \Theta}{\partial x^i} +  \frac{\partial^2 \bar{f}}{\partial(\ln p)^2} \mathcal{H}\Theta + \frac{\partial \bar{f}}{\partial (\ln p)} \left( \frac{\partial \Phi}{\partial \eta} - \hat{p}^i \frac{\partial \Psi}{\partial x^i} \right) \\
	&= - \frac{\partial \bar{f}}{\partial (\ln p)} \biggl[ \Theta' + \hat{p}^i \partial_i \Theta - \Phi' + \hat{p}^i \partial_i \Psi \biggr], \label{eqn:boltzmann_equation_collisionless_first}
\end{align}
after some algebra. There are four terms in \eqref{eqn:boltzmann_equation_collisionless_first}. The first two naturally arise from the free-streaming of temperature anisotropies. The other two in brackets account for the changes in photon energy caused by gravitational perturbations. They are equal to $-p^{-1}(d \ln(ap)/d\eta)$, where the \textit{comoving} energy $ap$ is constant in the background metric.

\subsubsection*{Thomson scattering} \label{section:thompson_scattering}

$e^- + \gamma \leftrightarrow e^- + \gamma$

$E_e (\vv{q}) = m_e + \frac{1}{2m_e} q^2$

Proper time of electrons $\eta_e$ (suppressing $\eta$ and $\vv{x}$ dependence for now)
\begin{align}
	\left. \frac{df}{d\eta_e} \right|_\text{scattering} = \bar{n}_e \int d^2 \hat{\vv{p}}_{in} \; \frac{d\sigma}{d\Omega} \left[ f(p,\hat{\vv{p}}_{in}) - f(p,\hat{\vv{p}})\right]
\end{align}
First - 'in' , second - 'out'.

\begin{align}
	\frac{d\sigma}{d\Omega} = \frac{3\sigma_T}{16\pi} (1 + \cos^2 \theta)
\end{align}
where $\cos\theta = \hat{\vv{p}} \cdot \hat{\vv{p}}_{in}$.

Account for bulk velocity $\vv{v}_b$ of baryons. Out modes integrates easily. $f = \bar{f} - (\partial\bar{f})/(\partial\ln p) \Theta$.
\begin{align}
	\left. \frac{df}{d\eta} \right|_\text{scattering} = a \bar{n}_e \sigma_T \frac{d\bar{f}}{d(\ln p)}  \left[ \Theta(\hat{\vv{p}}) - \frac{3}{16\pi} \int d^2 \hat{\vv{n}} \; \left[ \Theta(\hat{\vv{n}}) \left( 1 + (\hat{\vv{p}} \cdot \hat{\vv{n}})^2 \right) \right] - \hat{\vv{p}} \cdot \vv{v}_b \right]
\end{align}

Full Photon Boltzmann equation.
\begin{align}
	\Theta' + (\hat{\vv{p}} \cdot \nabla) \Theta &- \Phi' +(\hat{\vv{p}} \cdot \nabla) \Psi  \nonumber \\	
	&= a \bar{n}_e \sigma_T \left[ -\Theta + \frac{3}{16\pi} \int d^2 \hat{\vv{n}} \; \left[ \Theta(\hat{\vv{n}}) \left( 1 + (\hat{\vv{p}} \cdot \hat{\vv{n}})^2 \right) \right] + \hat{\vv{p}} \cdot \vv{v}_b \right] 
\end{align}

Optical depth
\begin{align}
	\tau(\eta) := \int_\eta^{\eta_0} d\eta'\; \left[ a(\eta') \bar{n}_e (\eta') \sigma_T \right]
\end{align}

$\tau' = -a \bar{n}_e \sigma_T$, visibility function $g:=-\tau'e^{-\tau}$

\begin{align}
	\frac{d}{d\eta} \left[ e^{-\tau} (\Theta + \Psi) \right] &= e^{-\tau}(\Psi' + \Phi') + g \left[ \frac{3}{16\pi} \int d^2\hat{\vv{n}} \; \left[ \Theta(\hat{\vv{n}}) (1 + (\hat{\vv{p}} \cdot \hat{\vv{n}})^2) \right] + \hat{\vv{p}} \cdot \vv{v}_b \right] \\
	&=: S(\eta, \vv{x}, \hat{\vv{p}})
\end{align}

Line of sight solution
\begin{align}
	\Theta(\eta_0, \vv{x}_0, \hat{\vv{p}}) + \Psi(\eta_0, \vv{x}_0) = \int_0^{\eta_0} d\eta' \; S(\eta', \vv{x}_0 - (\eta_0 -\eta') \hat{\vv{p}}, \hat{\vv{p}})
\end{align}
since $\tau(\eta_0)=0$, $\tau(0) = \infty$.

From line of sight solution, assume $g(\eta)\approx g(\eta_*)\delta(\eta-\eta_*)$ and no reionisation, so $e^{-\tau}$ is 0 before recombination $\eta=\eta_*$ and 1 afterwards. Also assume no anisotropic stress. Then the integral term becomes monopole $\Theta_0 = (1/4\pi)\int d^2\hat{\vv{n}} \;\Theta(\hat{\vv{n}})$ (will see later). In such case,

\begin{align}
	&\Theta(\eta_0, \vv{x}_0, \hat{\vv{p}}) + \Psi(\eta_0, \vv{x}_0) \nonumber \\
	&\qquad \approx \Theta_0(\eta_*, \vv{x}_*) + \Psi(\eta_*, \vv{x}_*) + \hat{\vv{p}}\cdot \vv{v}_b (\eta_*, \vv{x}_*) + \int_{\eta_*}^{\eta_0} d\eta' \; (\Psi' + \Phi')(\eta', \vv{x}_0-(\eta_0-\eta')\hat{\vv{p}}) \label{eqn:line_of_sight_solution_approximated}
\end{align}
where $\vv{x}_* = \vv{x}_0-(\eta_0-\eta_*)\hat{\vv{p}}$ is the location of LSS. Contributions: monopole, gravitational well to move out of, doppler effect from electrons at scattering, and ISW effect. Sum of monopole + psi term: the Sachs-Wolfe contribution.


Work in Fourier space. Define $\mu := \hat{\vv{p}}\cdot\hat{\vv{k}}$. Scalar perturbation -> axisymmetric $\Theta$ to first order. $\Theta(\eta,\vv{k},\hat{\vv{p}}) = \Theta(\eta,\vv{k},\mu)$. Define $l$ space quantities
\begin{align}
	\Theta_l (\eta,\vv{k}) :=& i^l \int \frac{d^2 \hat{\vv{p}}}{4\pi} \; P_l (\hat{\vv{k}} \cdot \hat{\vv{p}}) \Theta(\eta, \vv{k}, \hat{\vv{p}}) \\
	&= i^l \int^1_{-1} \frac{d\mu}{2} \; P_l (\mu) \Theta(\eta, \vv{k}, \mu)
\end{align}

Legendre polynomials orthogonality
\begin{align}
	\int_{-1}^{1} \frac{d\mu}{2} \; P_{l_1}(\mu) P_{l_2}(\mu) = \frac{1}{2l_1 + 1} \delta_{l_1 l_2}
\end{align}

\begin{align}
	\Theta(\eta,\vv{k},\mu) = \sum_{l=0}^{\infty} (-i)^l (2l+1) P_l (\mu) \Theta_l (\eta, \vv{k})
\end{align}
(Note that some literature defines theta as $(2l+1)\Theta_l$ in our convention)

Monopole is a simple angle average $\Theta_0 = (1/4\pi) \int d^2\hat{\vv{n}} \; \Theta(\hat{\vv{p}})$. Dipole $\Theta_1 = (i/4\pi) \int d^2\hat{\vv{n}} \; (\hat{\vv{k}} \cdot \hat{\vv{p}}) \Theta(\hat{\vv{p}})$

Using $P_2(\mu) \ (1/2)(3\mu^2 -1)$ and after some calculations
\begin{align}
	\frac{3}{16\pi} \int d^2\hat{\vv{n}} \; \left[ \Theta(\eta,\vv{k},\hat{\vv{n}}) (1 + (\hat{\vv{p}} \cdot \hat{\vv{n}})^2) \right] = \Theta_0 (\eta,\vv{k}) + \frac{1}{2} P_2 (\hat{\vv{k}} \cdot \hat{\vv{p}}) \Theta_2(\eta, \vv{k}) 
\end{align}

Boltzmann equation becomes (dropping $\eta$ and $\vv{k}$ dependence)
\begin{align}
	\Theta'(\mu) + ik\mu \Theta(\mu) - \Phi' + ik\mu\Psi = -\tau' \left[ - \Theta(\mu) + \Theta_0 + \frac{1}{2}P_2(\mu)\Theta_2 + ik\mu v_b \right]
\end{align}
where $\vv{v}_b = \nabla v_b$.

Noting $P_0(\mu) = 1$ and $P_1(\mu)=\mu$,
\begin{align}
	&\sum_{l=0}^{\infty} (-i)^l (2l+1) P_l (\mu) \left[ \Theta'_l + ik\mu \Theta_l \right] \nonumber\\
	&\quad= \sum_{l=0}^{\infty} (-i)^l (2l+1) P_l (\mu) \left[ \tau' \Theta_l + \delta_{l0} (\Phi' - \tau'\Theta_0) + \delta_{l1} (\frac{1}{3}k\Psi + \frac{1}{3}k\tau' v_b) + \delta_{l2}(\frac{1}{10}\tau' \Theta_2) \right]
\end{align}
The $\mu$ term on the left hand side is troublesome. Use the recursion relation for Legendre polynomials
\begin{align}
	(2l+1)\mu P_l(\mu) = (l+1) P_{l+1} (\mu) + l P_{l-1} (\mu)
\end{align}
Now the equation splits to each $l$ thanks to the orthogonality of Legendre polynomials.
\begin{align}
	&\Theta'_l - \frac{l}{2l+1} k\Theta_{l-1} + \frac{l+1}{2l+1} k\Theta_{l+1}  \nonumber\\
	&\qquad= \tau' \Theta_l + \delta_{l0} (\Phi' - \tau' \Theta_0) + \delta_{l1} (\frac{1}{3} k\Psi + \frac{1}{3} k\tau' v_b) + \delta_{l2}(\frac{1}{10}\tau' \Theta_2)
\end{align}

\begin{align}
	\Theta'_0 + k\Theta_1 &= \Phi' \\
	\Theta'_1 - \frac{1}{3} k\Theta_0 +\frac{2}{3} k\Theta_2 &= \frac{1}{3} k \Psi + \tau'(\Theta_1 + \frac{1}{3} k v_b)
\end{align}


Note the energy-momentum tensor
\begin{align}
	T^{\mu\nu} = \int \frac{d^3 \vv{p}}{E(\vv{p})} f(\vv{p}) p^i p^j
\end{align}
in the rest frame. Note the integration measure is Lorentz invariant. Have $\delta_\gamma = 4\Theta_0$, $v_\gamma = (-3i/k) \Theta_1$, and $\Pi_\gamma = -3\Theta_2$.



\subsubsection*{Line of sight solution} \label{section:line_of_sight_solution}
\subsection{Boltzmann hierarchy} \label{section:boltzmann_hierarchy}


\begin{itemize}
	\item Define photon anisotropy using its distribution function
	\item Boltzmann equation and Thompson scattering cross-section (quote)
	\item Decomposition using multipole moments
	\item Quote CAMB papers etc for detailed computation. Don't derive Boltzmann hierarchy!
	\item Mention that numerical solutions gives us transfer functions, defined as $\Delta_l (k) = \Theta_l(k,\eta_0) / \Phi(k,\eta_i)$
\end{itemize}


\subsection{Late-time anisotropy}

TODO: organise the following into sections

We have seen how to evolve perturbations in the photon distribution function $\Theta_l(k,\eta)$ from given initial conditions. In particular, the radiation transfer function defined as $\Delta_l(k) := \Theta(k,\eta_0) / \Phi(k,\eta_i) $ encapsulates all the relevant information about CMB photons' time evolution, from early times $\eta=\eta_i$ until now $\eta=\eta_0$. In this section, we derive the relation between perturbations $\Theta(k,\eta_0)$ and the CMB anisotropy observed today.

We observe the CMB at a fixed point in spacetime: here ($\vv{x}=\vv{x}_0$) and now ($\eta=\eta_0$). Small variations in time and location have completely negligible effects given the Hubble scale today. Anisotropy in the CMB temperature we observe hence takes the form
\begin{align}
	\left( \frac{\Delta T}{T} \right) (\hat{\vv{p}}) = \Theta(\vv{x}_0, \hat{\vv{p}}, \eta_0).
\end{align}
The vector $\hat{\vv{p}}$ relates to which direction in the sky we point our telescopes. Observed data lie on a two-dimensional sphere. A Fourier series equivalent for such function is the spherical harmonic expansion;
\begin{align}
	\Theta(\vv{x}, \hat{\vv{p}}, \eta) = \sum_{l,m} a_{lm}(\vv{x},\eta) Y_{lm}(\hat{\vv{p}}). 
\end{align} 
In physical terms, the spherical harmonics $Y_{lm}(\hat{\vv{p})}$ are joint eigenstate of angular momentum operators $\hat{L}^2$ and $\hat{L}_3$, with associated qunatum numbers $l$ and $m$, respecitvely. In mathematical terms, they form a basis of harmonic (vanishing Laplacian) polynomials of degree $l$ in three dimensions, restricted to a sphere. For each $l$, $m$ may take one of the $2l+1$ values in ${-l,-l+1,\cdots,l}$. Note that $Y_{lm}$s are orthonormal by construction:
\begin{align}
	\int d^2\hat{\vv{p}} \; Y^*_{lm} (\hat{\vv{p}}) Y_{l'm'} (\hat{\vv{p}}) = \delta_{ll'} \delta_{mm'}. \label{eqn:spherical_harmonic_orthonormality}
\end{align}
Another useful identity is the addition theorem for spherical harmonics given by
\begin{align}
	P_l (\hat{\vv{k}} \cdot \hat{\vv{p}}) = \frac{4\pi}{2l+1} \sum_{m=-l}^{l} Y^*_{lm} (\hat{\vv{k}}) Y_{lm} (\hat{\vv{p}}), \label{eqn:harmonic_addition_theorem}
\end{align}
where $P_l(\mu)$ are Legendre polynomials.
	
As long as sufficiently many multipoles $l$ are included, spherical harmonic coefficients $a_{lm}$ contain full information of the original function. Thanks to orthonormality, the spherical multipole coefficients can be computed using
\begin{align}
	a_{lm} (\vv{x},\eta) &= \int d^2\hat{\vv{p}} \; \Theta(\vv{x}, \hat{\vv{p}},\eta) Y^*_{lm} (\hat{\vv{p}}). \\
	&= \int \frac{d^3 k}{(2\pi)^3} e^{i\vv{k}\cdot\vv{x}} \int d^2\hat{\vv{p}} \; \Theta(\vv{k},\hat{\vv{p}},\eta) Y^*_{lm}(\hat{\vv{p}}). \label{eqn:alm_derivation_alm_using_theta}
\end{align}

Note that the evolution equation for perturbations (TODO: add refs to equations here) $\Theta(\vv{k},\hat{\vv{p}},\eta)$ here only depends on $k=\|\vv{k}\|$, the angle $\mu = \hat{\vv{k}} \cdot \hat{\vv{p}}$, and time $\eta$. The rest are determined from initial conditions. The following ratio therefore only depends on $k$, $\mu$ and $\eta$ as well;
\begin{align}
	\frac{\Theta(\vv{k},\hat{\vv{p}},\eta)}{\Phi(k,\eta_i)} &= f(k, \mu, \eta) \\
	&= \sum_l (-i)^l (2l+1) P_l(\mu) \frac{\Theta_l(k,\eta)}{\Phi(k,\eta_i)} \label{eqn:alm_derivation_legendre_expansion}\\ 
	&= \sum_l (-i)^l (2l+1) P_l(\mu) \Delta_l(k,\eta),
\end{align}
where we expanded using Legendre polynomials in \eqref{eqn:alm_derivation_legendre_expansion} and used the definition of transfer functions. Substituting this into \eqref{eqn:alm_derivation_alm_using_theta},
\begin{align}
	a_{lm}(\vv{x},\eta) &= \int \frac{d^3\vv{k}}{(2\pi)^3} e^{i\vv{k}\cdot\vv{x}} \Phi(k, \eta_i) \int d^2\hat{\vv{p}} \; \frac{\Theta(\vv{k},\hat{\vv{p}},\eta)}{\Phi(k, \eta_i)} Y^*_{lm}(\hat{\vv{p}}) \\
	&= \int \frac{d^3\vv{k}}{(2\pi)^3} e^{i\vv{k}\cdot\vv{x}} \Phi(k, \eta_i) \int d^2\hat{\vv{p}} \; \sum_{l'} \left[ (-i)^{l'} (2l'+1) P_{l'}(\hat{\vv{k}}\cdot\hat{\vv{p}}) \Delta_{l'}(k,\eta) Y^*_{lm}(\hat{\vv{p}}) \right] \\
	&= \int \frac{d^3\vv{k}}{(2\pi)^3} \left[ e^{i\vv{k}\cdot\vv{x}} \Phi(k, \eta_i) (-i)^{l} (2l+1) \Delta_{l}(k,\eta) Y^*_{lm}(\hat{\vv{k}}) \right].
\end{align}
For the last line, we expanded $P_{l'}$ using the addition theorem \eqref{eqn:harmonic_addition_theorem} and performed the $\int d\hat{\vv{p}}$ integral. Orthonormality \eqref{eqn:spherical_harmonic_orthonormality} forces $l'=l$ and simplifies the summation over $l'$.

Setting $\vv{x}=\vv{x_0}$ to be $\vv{0}$ and choosing $\eta=\eta_0$, we obtain a formula for the observed temperature anisotropy:
\begin{align}
	a_{lm} = 4\pi (-i)^l \int \frac{d^3\vv{k}}{(2\pi)^3} \Phi(k) \Delta_{l}(k) Y^*_{lm}(\hat{\vv{k}}).
\end{align}

\subsection{CMB power spectrum}

\begin{align}
	\langle \Phi (\vv{k}) \Phi (\vv{k}') \rangle = (2\pi)^3 \delta^{(3)} (\vv{k} - \vv{k}') P_\Phi (k)
\end{align}


\begin{align}
	\langle a^*_{lm} a_{l'm'} \rangle &= (4\pi)^2 i^{l-l'} \int \frac{d^3\vv{k}}{(2\pi)^3} \frac{d^3\vv{k'}}{(2\pi)^3} \langle \Phi (\vv{k}) \Phi (\vv{k}') \rangle (\vv{k}') \Delta_l (\vv{k}) \Delta_{l'} (\vv{k'}) Y_{lm}(\hat{\vv{k}}) Y^*_{l'm'}(\hat{\vv{k}'})  \\
	&= (4\pi)^2 i^{l-l'} \int \frac{d^3\vv{k}}{(2\pi)^3} \Delta_l (\vv{k}) \Delta_{l'} (\vv{k}) P_\Phi(k) Y_{lm}(\hat{\vv{k}}) Y^*_{l'm'}(\hat{\vv{k}}) \\
	&= \frac{2}{\pi} i^{l-l'} \int dk \; k^2  \Delta_l (\vv{k}) \Delta_{l'} (\vv{k}) P(k) \delta_{ll'} \delta_{mm'} \\
	&= \frac{2}{\pi} \int dk \; k^2 |\Delta_l(k)|^2 P_\Phi (k)
\end{align}


Borrow Planck 2018 analysis picture. 
