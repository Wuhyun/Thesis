%!TEX root = ../thesis.tex
%*******************************************************************************
%****************************** Second Chapter *********************************
%*******************************************************************************

\chapter{Cosmic Microwave Background Anisotropy}

\ifpdf
    \graphicspath{{Chapter2/Figs/Raster/}{Chapter2/Figs/PDF/}{Chapter2/Figs/}}
\else
    \graphicspath{{Chapter2/Figs/Vector/}{Chapter2/Figs/}}
\fi

Cosmic Microwave Background is like a trove full of precious cosmological information.
(TODO: discovery led to confirmation of big bang. Anisotropy constrains cosmological parameters, almost by itself. success of LCDM. Bispectrum. Lensing map and secondaries like tSZ, kSZ...)



\begin{itemize}
	\item Explain recombination and CMB, without talking about detailed thermal history 
	\item CMB perfectly follows blackbody spectrum. Borrow a figure of intensity vs wavelength from Planck
	\item Existence of temperature anisotropy. Maps from COBE, WMAP and Planck
	\item Explain goal: obtain theoretical derivations for CMB anisotropy and introduce statistical formulation for comparing with measurements.	
\end{itemize}

\section{The inhomogeneous universe}

\subsection{Metric perturbations}

Recall that the FLRW metric is given using conformal time by
\begin{align}
	ds^2 \bar{a}(t)^2 (-d\tau^2 + d\vv{x}^2),
\end{align}
where a bar is used to denote quantities computed from the homogeneous background universe in Chapter \ref{chapter:introduction}. We write perturbations around this background metric as follows.
\begin{align}
	ds^2 = \bar{a}(t)^2 \left[ -(1+2A) dt^2 + 2B_i \; d\tau dx^i + (\delta_{ij} + h_{ij}) dx^i dx^j \right].
\end{align}
The spatial indices $i,j,\cdots$ are lowered and raised using $\delta_{ij}$ here. Note that the scale factor has not been perturbed, since any variation of it can be absorbed into other perturbative variables. There are 1, 3 and 6 degrees of freedom coming from $A$, $B_i$ and $h_{ij}$, respectively, adding up to 10 as expected from a 3+1 dimensional spacetime.

We further extract the divergence part from $B_i$ and $h_{ij}$, as well as the trace of $h_{ij}$. Writing $V$ and $T$ to denote vector and tensor quantities, 
\begin{align}
	B_i &= \partial_i B + B_i^V \\
	h_{ij} &= 2C \delta_{ij} + 2(\partial_i \partial_j - \frac{1}{3} \nabla^2 ) E + (\partial_i E_j^V + \partial_j E_i^V). + 2E^T_{ij}.
\end{align}
The variables are chosen such that $B_i^V$ has vanishing divergence, and $h_{ij}^T$ is traceless and transverse.

The Scalar-Vector-Tensor (SVT) theorem states that to linear order in the perturbations, these modes decouple and evolve in three independent groups: scalars ($A,B,C,E$), vectors ($B_i^V,E_i^V$), and tensors ($E_{ij}^T$). Each of these contain 4, 4, and 2 degrees of freedom, respectively, again totalling to 10 as required. In this section, we are only interested in scalar perturbations which generates temperature anisotropy and E-mode polarisation in the CMB.

Keeping only the scalar modes, the perturbed metric becomes
\begin{align}
	ds^2 = \bar{a}(\tau)^2 \left[ -(1+2A) dt^2 + 2 \partial_i B \; d\tau dx^i + \left( (1+2C)\delta_{ij} + 2(\partial_i \partial_j - \frac{1}{3} \delta_{ij} \nabla^2 ) E \right) dx^i dx^j \right].
\end{align}

Coordinate transformation are \textit{gauge} symmetries in General Relativity; they correspond to redundancies in our mathematical representation of the system. Redefining the coordinate variables may change how the theory looks like, but all physical results derived the new set are equivalent to the original theory and are just a coordinate transformation away.

In the perturbation theory under consideration, we are free to make gauge transformations $x^\mu \rightarrow \tilde{x}^\mu = x^\mu + \xi^\mu$.

metric transforms.

There is a freedom to impose extra conditions.


\begin{itemize}
	\item Gauge problem. use Newtonian gauge
	\item Perturbed Einstein equations
	\item Definition of $\zeta$ and its conservation in superhorizon scales
\end{itemize}

\subsection{Boltzmann Equations}
\begin{itemize}
	\item Define photon anisotropy using its distribution function
	\item Boltzmann equation and Thompson scattering cross-section (quote)
	\item Decomposition using multipole moments
	\item Quote CAMB papers etc for detailed computation. Don't derive Boltzmann hierarchy!
	\item Mention that numerical solutions gives us transfer functions, defined as $\Delta_l (k) = \Theta_l(k,\eta_0) / \Phi(k,\eta_i)$
\end{itemize}

\subsection{CMB anisotropy}

TODO: organise the following into sections

We have seen how to evolve perturbations in the photon distribution function $\Theta_l(k,\eta)$ from given initial conditions. In particular, the radiation transfer function $\Delta_l(k) := \Theta(k,\eta_0) / \Phi(k,\eta_i) $ encapsulates all the relevant information about CMB photons' time evolution, from early times $\eta=\eta_i$ until now $\eta=\eta_0$. In this section, we derive the relation between perturbations $\Theta(k,\eta_0)$ and the CMB anisotropy observed today.

We observe the CMB at a fixed point in spacetime: here ($\vv{x}=\vv{x}_0$) and now ($\eta=\eta_0$). Small variations in time and location have completely negligible effects given the Hubble scale today. Anisotropy in the CMB temperature we observe hence takes the form
\begin{align}
	\left( \frac{\Delta T}{T} \right) (\hat{\vv{p}}) = \Theta(\vv{x}_0, \hat{\vv{p}}, \eta_0).
\end{align}
The vector $\hat{\vv{p}}$ relates to which direction in the sky we point our telescopes. Observed data lie on a two-dimensional sphere. A Fourier series equivalent for such function is the spherical harmonic expansion;
\begin{align}
	\Theta(\vv{x}, \hat{\vv{p}}, \eta) = \sum_{l,m} a_{lm}(\vv{x},\eta) Y_{lm}(\hat{\vv{p}}). 
\end{align} 
In physical terms, the spherical harmonics $Y_{lm}(\hat{\vv{p})}$ are joint eigenstate of angular momentum operators $\hat{L}^2$ and $\hat{L}_3$, with associated qunatum numbers $l$ and $m$, respecitvely. In mathematical terms, they form a basis of harmonic (vanishing Laplacian) polynomials of degree $l$ in three dimensions, restricted to a sphere. For each $l$, $m$ may take one of the $2l+1$ values in ${-l,-l+1,\cdots,l}$. Note that $Y_{lm}$s are orthonormal by construction:
\begin{align}
	\int d^2\hat{\vv{p}} \; Y^*_{lm} (\hat{\vv{p}}) Y_{l'm'} (\hat{\vv{p}}) = \delta_{ll'} \delta_{mm'}. \label{eqn:spherical_harmonic_orthonormality}
\end{align}
Another useful identity is the addition theorem for spherical harmonics given by
\begin{align}
	P_l (\hat{\vv{k}} \cdot \hat{\vv{p}}) = \frac{4\pi}{2l+1} \sum_{m=-l}^{l} Y^*_{lm} (\hat{\vv{k}}) Y_{lm} (\hat{\vv{p}}), \label{eqn:harmonic_addition_theorem}
\end{align}
where $P_l(\mu)$ are Legendre polynomials.
	
As long as sufficiently many multipoles $l$ are included, spherical harmonic coefficients $a_{lm}$ contain full information of the original function. Thanks to orthonormality, the harmonic coefficients can be computed using
\begin{align}
	a_{lm} (\vv{x},\eta) &= \int d^2\hat{\vv{p}} \; \Theta(\vv{x}, \hat{\vv{p}},\eta) Y^*_{lm} (\hat{\vv{p}}). \\
	&= \int \frac{d^3 k}{(2\pi)^3} e^{i\vv{k}\cdot\vv{x}} \int d^2\hat{\vv{p}} \; \Theta(\vv{k},\hat{\vv{p}},\eta) Y^*_{lm}(\hat{\vv{p}}). \label{eqn:alm_derivation_alm_using_theta}
\end{align}

Note that the evolution equation for perturbations (TODO: add refs to equations here) $\Theta(\vv{k},\hat{\vv{p}},\eta)$ here only depends on $k=\|\vv{k}\|$, the angle $\mu = \hat{\vv{k}} \cdot \hat{\vv{p}}$, and time $\eta$. The rest are determined from initial conditions. The following ratio therefore only depends on $k$, $\mu$ and $\eta$ as well;
\begin{align}
	\frac{\Theta(\vv{k},\hat{\vv{p}},\eta)}{\Phi(k,\eta_i)} &= f(k, \mu, \eta) \\
	&= \sum_l (-i)^l (2l+1) P_l(\mu) \frac{\Theta_l(k,\eta)}{\Phi(k,\eta_i)} \label{eqn:alm_derivation_legendre_expansion}\\ 
	&= \sum_l (-i)^l (2l+1) P_l(\mu) \Delta_l(k,\eta),
\end{align}
where we expanded using Legendre polynomials in \eqref{eqn:alm_derivation_legendre_expansion} and used the definition of transfer functions. Substituting this into \eqref{eqn:alm_derivation_alm_using_theta},
\begin{align}
	a_{lm}(\vv{x},\eta) &= \int \frac{d^3\vv{k}}{(2\pi)^3} e^{i\vv{k}\cdot\vv{x}} \Phi(k, \eta_i) \int d^2\hat{\vv{p}} \; \frac{\Theta(\vv{k},\hat{\vv{p}},\eta)}{\Phi(k, \eta_i)} Y^*_{lm}(\hat{\vv{p}}) \\
	&= \int \frac{d^3\vv{k}}{(2\pi)^3} e^{i\vv{k}\cdot\vv{x}} \Phi(k, \eta_i) \int d^2\hat{\vv{p}} \; \sum_{l'} \left[ (-i)^{l'} (2l'+1) P_{l'}(\hat{\vv{k}}\cdot\hat{\vv{p}}) \Delta_{l'}(k,\eta) Y^*_{lm}(\hat{\vv{p}}) \right] \\
	&= \int \frac{d^3\vv{k}}{(2\pi)^3} \left[ e^{i\vv{k}\cdot\vv{x}} \Phi(k, \eta_i) (-i)^{l} (2l+1) \Delta_{l}(k,\eta) Y^*_{lm}(\hat{\vv{k}}) \right].
\end{align}
For the last line, we expanded $P_{l'}$ using the addition theorem \eqref{eqn:harmonic_addition_theorem} and performed the $\int d\hat{\vv{p}}$ integral. Orthonormality \eqref{eqn:spherical_harmonic_orthonormality} forces $l'=l$ and simplifies the summation over $l'$.

Setting $\vv{x}=\vv{x_0}$ to be $\vv{0}$ and choosing $\eta=\eta_0$, we obtain a formula for the observed temperature anisotropy:
\begin{align}
	a_{lm} = 4\pi (-i)^l \int \frac{d^3\vv{k}}{(2\pi)^3} \Phi(k) \Delta_{l}(k) Y^*_{lm}(\hat{\vv{k}}).
\end{align}

\subsection{CMB power spectrum}

\begin{align}
	\langle \Phi (\vv{k}) \Phi (\vv{k}') \rangle = (2\pi)^3 \delta^{(3)} (\vv{k} - \vv{k}') P_\phi (k)
\end{align}


\begin{align}
	\langle a^*_{lm} a_{l'm'} \rangle &= (4\pi)^2 i^{l-l'} \int \frac{d^3\vv{k}}{(2\pi)^3} \frac{d^3\vv{k'}}{(2\pi)^3} \langle \Phi (\vv{k}) \Phi (\vv{k}') \rangle (\vv{k}') \Delta_l (\vv{k}) \Delta_{l'} (\vv{k'}) Y_{lm}(\hat{\vv{k}}) Y^*_{l'm'}(\hat{\vv{k}'})  \\
	&= (4\pi)^2 i^{l-l'} \int \frac{d^3\vv{k}}{(2\pi)^3} \Delta_l (\vv{k}) \Delta_{l'} (\vv{k}) P(k) Y_{lm}(\hat{\vv{k}}) Y^*_{l'm'}(\hat{\vv{k}}) \\
	&= \frac{2}{\pi} i^{l-l'} \int dk \; k^2  \Delta_l (\vv{k}) \Delta_{l'} (\vv{k}) P(k) \delta_{ll'} \delta_{mm'} \\
	&= \frac{2}{\pi} \int dk \; k^2 |\Delta_l(k)|^2 P_\Phi (k)
\end{align}






\begin{itemize}
	\item Derive form for $C_l$'s. (Written)
	\item Borrow Planck 2018 analysis picture. 
\end{itemize}
