%!TEX root = ../thesis.tex
%*******************************************************************************
%****************************** Second Chapter *********************************
%*******************************************************************************

\chapter{Cosmic Microwave Background Anisotropy}

\ifpdf
    \graphicspath{{Chapter2/Figs/Raster/}{Chapter2/Figs/PDF/}{Chapter2/Figs/}}
\else
    \graphicspath{{Chapter2/Figs/Vector/}{Chapter2/Figs/}}
\fi


\begin{itemize}
	\item Explain recombination and CMB, without talking about detailed thermal history 
	\item CMB perfectly follows blackbody spectrum. Borrow a figure of intensity vs wavelength from Planck
	\item Existence of temperature anisotropy. Maps from COBE, WMAP and Planck
	\item Explain goal: obtain theoretical derivations for CMB anisotropy and introduce statistical formulation for comparing with measurements.	
\end{itemize}

\section{The inhomogeneous universe}

\subsection{Perturbation theory}
\begin{itemize}
	\item Definitions for perturbation theory
	\item SVT theorem, only consider scalar perturbations
	\item Gauge problem. use Newtonian gauge
	\item Perturbed Einstein equations
	\item Definition of $\zeta$ and its conservation in superhorizon scales
\end{itemize}

\subsection{Boltzmann Equations}
\begin{itemize}
	\item Define photon anisotropy using its distribution function
	\item Boltzmann equation and Thompson scattering cross-section (quote)
	\item Decomposition using multipole moments
	\item Quote CAMB papers etc for detailed computation. Don't derive Boltzmann hierarchy!
	\item Mention that numerical solutions gives us transfer functions, defined as $\Delta_l (k) = \Theta_l(k,\eta_0) / \Phi(k,\eta_i)$
\end{itemize}

\subsection{CMB anisotropy}
\begin{itemize}
	\item Late time $a_{lm}$s. Derive expression using transfer functions. (Written)
	\item Derive form for $C_l$'s. (Written)
	\item Borrow Planck 2018 analysis picture. 
\end{itemize}
