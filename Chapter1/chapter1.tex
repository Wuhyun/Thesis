%!TEX root = ../thesis.tex
%*******************************************************************************
%*********************************** First Chapter *****************************
%*******************************************************************************

\chapter{Introduction}

\ifpdf
    \graphicspath{{Chapter1/Figs/Raster/}{Chapter1/Figs/PDF/}{Chapter1/Figs/}}
\else
    \graphicspath{{Chapter1/Figs/Vector/}{Chapter1/Figs/}}
\fi


%********************************** %First Section  **************************************
\section{From background to foreground}

Radio astronomers Arno Penzias and Robert Wilson were calibrating their 50-feet-long horn antenna when they found a mysterious background noise. The measurements were independent of time and location in the sky, and persisted after the removal of various potential contaminants. After the theoretical work of Robert Dicke, Jim Peebles, and David Wilkinson was brought forward \cite{Dicke1965}, Penzias and Wilson identified the noise as cosmic microwave background radiation (CMB): ancient light from the early universe reaching us after billions of years \cite{Penzias1965}. The discovery provided us with one of the most valuable probes of the physical universe, leading to major development in observational cosmology.

On the theoretical side, modern mathematical formulation of cosmology owes to Einstein's work on general relativity in 1915. Using his framework, Friedmann, Lemaître, Robertson, and Walker contributed to writing down the unique metric for spatially homogeneous and isotropic universe. The FLRW metric dictates growth of the universe from the Big Bang to present day. Such expansion of the universe was supported by Edwin Hubble's measurements of Cepheid variables and redshift (add year), as well as the aforementioned discovery of CMB. What is widely accepted to be the standard model of modern cosmology, the $\Lambda$CDM model, appeared only in the late 1990s. The six-parameter model assumes presence of cold dark matter and dark energy, in addition to baryons and radiation, as main contributors to the total energy density of the universe.

The recent Planck satellite mission has played a major role in estimating various cosmological parameters. Planck


CMB measures parameters. Seeded by initial conditions, feels gravity - backlight of the universe's aquarium. 

Numerous CMB experiments including Planck. Planck specifications, LCDM successfully explains measured anisotropy. Tight constraints on cosmological parameters, as well as inflationary scenarios.

(Primordial non-Gaussianity and bispectrum?)

Upcoming CMB experiments. Simons and CMB-stage 4.

The need for development of bispectrum routines suitable for new specifications...

\section{The homogenous universe}

\subsection*{General Relativity}

The revolutionary idea of Einstein was that spacetime is be bent by its contents. Physical distance in the curved spacetime is represented using a metric tensor $g$;
\begin{align}
	ds^2 \,=\, g_{\mu \nu} \; dx^\mu dx^\nu	,
\end{align}
where the Greek letters $\mu, \nu = 0,1,2,3$ denote time ($0$) and spatial ($1-3$) indices. Flat spacetime has metric $g_{\mu\nu} = \eta_{\mu\nu} = \text{diag}\{-1, 1, 1, 1\}$, known as the Minkowski metric. Throughout this thesis we adopt the sign convention $(-, +, +, +)$ and work in units where $c=1$. Unless specified otherwise, the Einstein summation convention is assumed. 

In curved spacetime, a comoving (free-falling) object may follow a non-trivial trajectory given by the geodesic equation;
\begin{align}
	\frac{d^2x^\mu}{ds^2} \,+\, \Gamma^\mu_{\nu \rho} \frac{dx^\nu}{ds} \frac{dx^\rho}{ds} \,=\, 0,
\end{align}
with $s$ an affine parameter parametrising the trajectory, and $\Gamma^\mu_{\nu\rho}$ the Christoffel symbol representing metric connection. Its values are given in terms of the metric tensor by
\begin{align}
	\Gamma^{\mu}_{\nu\rho} \,=\, \frac{1}{2}~ g^{\mu\sigma} \left( g_{\nu\sigma,\rho} \,+\, g_{\rho\sigma,\nu} \,-\, g_{\nu\rho,\sigma}  \right).
\end{align}
Here and for rest of this thesis, commas in subscripts $(\cdot)_{,\mu}$ denote partial derivatives with respect to given index $\mu$. $g^{\nu\sigma}$ is the inverse metric satisfying $g^{\mu\nu} g_{\nu\rho} \,=\, \delta^\nu_\rho$.


